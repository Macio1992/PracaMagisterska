\documentclass[11pt, a4paper]{article}
\usepackage{polski}
\usepackage[utf8]{inputenc}
\usepackage{listings}

\begin{document}
\lstset{language=C++}

\section{Szablony - definicja, zastosowania}
Szablony są jedną z głównych cech języka \emph{C++}. Dzięki nim możemy dostarczać generyczne typy i funkcje, bez kosztów czasu wykonania. Skupiają się na pisaniu kodu w sposób niezależny od konkretnego typu, dzięki czemu wspierają programowanie generyczne. \emph{C++} to bogaty język wspierający polimorficzne zachowania zarówno w czasie wykonania jak i kompilacji. W tym pierwszym używa hierarchii klas i wywołań funkcji wirtualnych by wspierać praktyki zorientowane obiektowo, gdzie wywoływana funkcja zależy od typu obiektu docelowego podczas czasu wykonania. Natomiast w czasie kompilacji szablony wspierają programowanie generyczne, gdzie wywoływana funkcja zależy od statycznego typu czasu kompilacji argumentów szablonu.

Polimorfizm czasu kompilacji był w języku od bardzo dawna. Polega on na dostarczeniu szablonu, który umożliwia kompilatorowi wygenerowanie kodu w czasie kompilacji.

Grają kluczową rolę w projektowaniu obecnych, znanych i popularnych bibliotek i systemów. Stanowią podstawę technik programowania w różnych dziedzinach, począwszy od konwencjonalnego programowania ogólnego przeznaczenia do oprogramowywania wbudowanych 
systemów bezpieczeństwa.

Szablon to coś w rodzaju przepisu, z którego translator \emph{C++} generuje deklaracje.

\begin{lstlisting}[frame=single]
template<typename T>
T kwadrat (T x) {
   return x * x;
}
\end{lstlisting}

Kod ten deklaruje rodzinę funkcji indeksowanych po parametrze typu. Można odnieść się do konkretnego członka tej rodziny przez zastosowanie konstrukcji \verb#kwadrat<int>#. Mówimy wtedy, że żądana jest specjalizacja szablonu dla funkcji \verb#kwadrat# z listą argumentów szablonu \verb#<int>#. Proces tworzenia specjalizacji nosi nazwę inicjalizacji szablonu, potocznie zwany inicjalizacją. Kompilator C++ stworzy stosowny odpowiednik definicji funkcji:

\begin{lstlisting}[frame=single]
int kwadrat(int x) {
   return x * x;
}
\end{lstlisting}

Argument typu \verb#int# jest podstawiony za parametr typu \verb#T#. Kod wynikowy jest sprawdzany pod względem typu, by zapewnić brak błędów wynikających z podmiany. Inicjalizacja szablonu jest wykonywana tylko raz dla danej specyfikacji nawet jeśli program zawiera jej wielokrotne żądania. 

W przeciwieństwie do języków takich jak \emph{Ada} czy \emph{System F}, lista argumentów szablonu może być pominięta z żądania inicjalizacji szablonu funkcji. Zazwyczaj, wartości parametrów szablonu są dedukowane.
\newline

\verb#double d = kwadrat(2.0);# \newline

Argument typu jest dedukowany na \verb#double#. Warto zauważyć, że odmiennie niż w językach takich jak \emph{Haskell} czy \emph{System F}, parametry szablonu w \emph{C++} nie są ograniczone względem typów.

Szablonów używa się do zmniejszania kar abstrakcji i zjawiska \emph{code bloat} w systemach wbudowanych w stopniu, który jest niepraktyczny w standardowych systemach obiektowych. Robi się to z dwóch powodów:

\begin{itemize}

\item Po pierwsze, inicjalizacja szablonu łączy informacje zarówno z definicji, jak i z kontekstu użycia. To oznacza, że pełna informacja zarówno z definicji jak i z wywołanych kontekstów (włączając w to informacje o typach) jest udostępniana generatorowi kodu. Dzisiejsze generatory kodu dobrze sobie radzą z używaniem tych informacji w celu zminimalizowania czasu wykonania i przestrzeni kodu. Różni się to od zwykłego przypadku w języku obiektowym, gdzie wywołujący i wywoływany są kompletnie oddzieleni przez interfejs, który zakłada pośrednie wywołania funkcji.

\item Po drugie, szablon w C++ jest zazwyczaj domyślnie tworzony tylko jeśli jest używany w sposób niezbędny dla semantyki programu, automatycznie minimalizując miejsce w pamięci, które wykorzystuje aplikacja. W przeciwieństwie do języka \emph{Ada} czy \emph{System F}, gdzie programista musi wyraźnie zarządzać inicjalizacjami.

\end{itemize}

\end{document}