\documentclass[11pt, a4paper]{article}
\usepackage{polski}
\usepackage[utf8]{inputenc}
\usepackage{listings}

\begin{document}

\subsection*{Wprowadzenie}

Każda szeroko używana biblioteka generyczna jest oparta na konceptach. Te koncepty można przedstawiać za pomocą specjalnie zaprojektowanych funkcji języka, tabel wymagań, komentarzy w kodzie, dokumentach projektu lub po prostu w głowach programistów. Lecz bez konceptów (formalnych czy nieformalnych), żaden kod generyczny by nie działał.

Podczas badań nad konceptami dochodzono do różnych wniosków. Zastanawiano się nad wpływem konceptów na niezmienniki (czy będą wzmocnione czy osłabną), opuszczaniem operatorów czy ograniczaniu przestrzeni definicji. Z tych rozważań wywnioskowano, że pisanie generycznej biblioteki wymaga wyczerpującego wyliczania przeładowanych operacji, wsparcia konceptów dla metaprogramowania szablonów i ogromnego bezpośredniego wsparcia języka dla konceptów.

Na szczęście, te podsumowanie mijają się z prawdą. Są po prostu złe. Przeważająca większość bibliotek generycznych, w tym oryginalna wersja biblioteki \emph{STL}\footnote{Standard Template Library}, została napisana bez wsparcia języka dla konceptów i używana bez mechanizmów wykonawczych. Podczas rozwoju bibliotek, ich autorzy nie rozważali setki różnych konceptów. Raczej, generyczne komponenty tych bibliotek zostały zapisane w małym zbiorze wyidealizowanych abstrakcji.

\addcontentsline{toc}{subsection}{Wprowadzenie}

\end{document}