\documentclass[11pt, a4paper]{article}
\usepackage{polski}
\usepackage[utf8]{inputenc}
\usepackage{listings}
\usepackage{standalone}

\begin{document}
\lstset{language=C++}

Pomysł ograniczania argumentów szablonów jest tak stary jak stare są same szablony. Ale dopiero z początkiem dwudziestego pierwszego wieku zaczęły się poważne prace nad projektem języka C++, aby zapewnić te możliwości. Te prace ostatecznie dały rezultat w postaci \emph{konceptów C++0x}. Rozwój tych funkcjonalności i ich wdrożenie do biblioteki standardowej \emph{C++} były głównymi tematami \emph{Komisji Standardu C++}\footnote{(ang. C++ Standards Committee) znana również pod nazwą "ISO JTC1/SC22/WG21". Składa się z akredytowanych ekspertów z krajów członkowskich, którzy są zainteresowani pracą nad C++} dla C++11. Te cechy zostały ostatecznie usunięte z powodu istotnych nierozwiązanych kwestii i bardziej rygorystycznego terminu publikacji.

W 2010 roku wznowiono prace nad konceptami (bez udziału komisji). Andrew Sutton i Bjarne Stroustrup opublikowali dokument omawiający jak zminimalizować ilość konceptów potrzebnych do określania części biblioteki standardowej, a grupa z Uniwersytetu w Indianie zainicjowała prace nad nową implementacją. Potem wspólnie z Alexem Stepanovem (twórcą \emph{biblioteki STL}\footnote{(ang. Standard Template Library)}) stworzyli raport, w którymi przedstawili w pełni ograniczone algorytmy biblioteki STL i zasugerowali projekt języka, który mógłby wyrazić te ograniczenia. Próbowano zaprojektować minimalny zestaw funkcji językowych, które umożliwiłyby użytkownikom ograniczanie szablonów. Z tych prób narodziło się rozszerzenie języka, zwane \emph{Concepts Lite}.

Koncepty nie zostały włączone w \emph{C++17}. Niektórzy członkowie komisji uważali że nie minęło wystarczająco dużo czasu od publikacji specyfikacji technicznej, żeby sprawdzić czy projekt jest wystarczająco dobry, a wielu z nich było niezdecydowanych.

\end{document}