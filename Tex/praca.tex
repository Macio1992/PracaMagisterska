\documentclass[11pt, a4paper]{article}
\usepackage{polski}
\usepackage[utf8]{inputenc}

\author{Maciej Zbierowski}
\title{Koncepty jako sposób ograniczania argumentów szablonu}

\linespread{1.3}
\usepackage{hyperref}
\usepackage{listings}
\usepackage{standalone}

\hypersetup{
	colorlinks=true,
	linkcolor=black,
	filecolor=magenta,
	urlcolor=cyan,
	pdftitle=(Praca Magisterska)
	bookmarks=true,
	pdfpagemode=FullScreen
}

\begin{document}
	\pagenumbering{gobble}
	\maketitle
	\newpage
	\tableofcontents
	\pagenumbering{arabic}
	\newpage
	
	\section*{Wstęp}

	\documentclass[11pt, a4paper]{article}
\usepackage{polski}
\usepackage[utf8]{inputenc}
\usepackage{listings}
\usepackage{standalone}

\begin{document}
\lstset{language=C++}

Pomysł ograniczania argumentów szablonów jest tak stary jak stare są same szablony. Ale dopiero z początkiem dwudziestego pierwszego wieku zaczęły się poważne prace nad projektem języka C++, aby zapewnić te możliwości. Te prace ostatecznie dały rezultat w postaci \emph{konceptów C++0x}. Rozwój tych funkcjonalności i ich wdrożenie do biblioteki standardowej \emph{C++} były głównymi tematami \emph{Komisji Standardu C++}\footnote{(ang. C++ Standards Committee) znana również pod nazwą "ISO JTC1/SC22/WG21". Składa się z akredytowanych ekspertów z krajów członkowskich, którzy są zainteresowani pracą nad C++} dla C++11. Te cechy zostały ostatecznie usunięte z powodu istotnych nierozwiązanych kwestii i bardziej rygorystycznego terminu publikacji.

W 2010 roku wznowiono prace nad konceptami (bez udziału komisji). Andrew Sutton i Bjarne Stroustrup opublikowali dokument omawiający jak zminimalizować ilość konceptów potrzebnych do określania części biblioteki standardowej, a grupa z Uniwersytetu w Indianie zainicjowała prace nad nową implementacją. Potem wspólnie z Alexem Stepanovem (twórcą \emph{biblioteki STL}\footnote{(ang. Standard Template Library)}) stworzyli raport, w którymi przedstawili w pełni ograniczone algorytmy biblioteki STL i zasugerowali projekt języka, który mógłby wyrazić te ograniczenia. Próbowano zaprojektować minimalny zestaw funkcji językowych, które umożliwiłyby użytkownikom ograniczanie szablonów. Z tych prób narodziło się rozszerzenie języka, zwane \emph{Concepts Lite}.

Koncepty nie zostały włączone w \emph{C++17}. Niektórzy członkowie komisji uważali że nie minęło wystarczająco dużo czasu od publikacji specyfikacji technicznej, żeby sprawdzić czy projekt jest wystarczająco dobry, a wielu z nich było niezdecydowanych.

\end{document}

	\addcontentsline{toc}{section}{Wstęp}
	
	\documentclass[11pt, a4paper]{article}
\usepackage{polski}
\usepackage[utf8]{inputenc}
\usepackage{listings}
\usepackage{standalone}

\begin{document}
\lstset{language=C++}

\documentclass[11pt, a4paper]{article}
\usepackage{polski}
\usepackage[utf8]{inputenc}
\usepackage{listings}
\usepackage{standalone}

\begin{document}
\lstset{language=C++}

Pomysł ograniczania argumentów szablonów jest tak stary jak stare są same szablony. Ale dopiero z początkiem dwudziestego pierwszego wieku zaczęły się poważne prace nad projektem języka C++, aby zapewnić te możliwości. Te prace ostatecznie dały rezultat w postaci \emph{konceptów C++0x}. Rozwój tych funkcjonalności i ich wdrożenie do biblioteki standardowej \emph{C++} były głównymi tematami \emph{Komisji Standardu C++}\footnote{(ang. C++ Standards Committee) znana również pod nazwą "ISO JTC1/SC22/WG21". Składa się z akredytowanych ekspertów z krajów członkowskich, którzy są zainteresowani pracą nad C++} dla C++11. Te cechy zostały ostatecznie usunięte z powodu istotnych nierozwiązanych kwestii i bardziej rygorystycznego terminu publikacji.

W 2010 roku wznowiono prace nad konceptami (bez udziału komisji). Andrew Sutton i Bjarne Stroustrup opublikowali dokument omawiający jak zminimalizować ilość konceptów potrzebnych do określania części biblioteki standardowej, a grupa z Uniwersytetu w Indianie zainicjowała prace nad nową implementacją. Potem wspólnie z Alexem Stepanovem (twórcą \emph{biblioteki STL}\footnote{(ang. Standard Template Library)}) stworzyli raport, w którymi przedstawili w pełni ograniczone algorytmy biblioteki STL i zasugerowali projekt języka, który mógłby wyrazić te ograniczenia. Próbowano zaprojektować minimalny zestaw funkcji językowych, które umożliwiłyby użytkownikom ograniczanie szablonów. Z tych prób narodziło się rozszerzenie języka, zwane \emph{Concepts Lite}.

Koncepty nie zostały włączone w \emph{C++17}. Niektórzy członkowie komisji uważali że nie minęło wystarczająco dużo czasu od publikacji specyfikacji technicznej, żeby sprawdzić czy projekt jest wystarczająco dobry, a wielu z nich było niezdecydowanych.

\end{document}

\documentclass[11pt, a4paper]{article}
\usepackage{polski}
\usepackage[utf8]{inputenc}
\usepackage{listings}

\begin{document}
\lstset{language=C++}

\subsection{Parametryzacja szablonów}

Parametry szablonu są specyfikowane na dwa sposoby:

\begin{enumerate}

\item \emph{parametry szablonu} – wyraźnie wspomniane jako parametry w deklaracji szablonu

\item \emph{nazwy zależne} - wywnioskowane z użycia parametrów w definicji szablonu

\end{enumerate}

W \emph{C++} nazwa nie może być użyta bez wcześniejszej deklaracji. To wymaga od użytkownika ostrożnego traktowania definicji szablonów. Np. w definicji funkcji \verb#kwadrat# nie ma widocznej deklaracji symbolu *. Jednak, podczas inicjalizacji szablonu \verb#kwadrat<int># kompilator może sprowadzić symbol * do (wbudowanego) operatora mnożenia dla wartości \verb#int#. Dla wywołania \verb#kwadrat(zespolona(2.0))#, operator * zostałby rozwiązany do (zdefiniowanego przez użytkownika) operatora mnożenia dla wartości \verb#zespolona#. Symbol * jest więc \emph{nazwą zależną} w definicji funkcji \verb#kwadrat#. Oznacza to, że jest to ukryty parametr definicji szablonu. Możemy uczynić z operacji mnożenia formalny parametr:

\begin{lstlisting}[frame=single]
template<typename Multiply, typename T>
T square(T x) {
   return Multiply() (x,x);
}
\end{lstlisting}

Pod-wyrażenie \verb#Multiply()# tworzy obiekt funkcji, który wprowadza operację mnożenia wartości typu \verb#T#. Pojęcie \emph{nazw zależnych} pomaga utrzymać liczbę jawnych argumentów.

\end{document}

\documentclass[11pt, a4paper]{article}
\usepackage{polski}
\usepackage[utf8]{inputenc}
\usepackage{listings}

\begin{document}
\lstset{language=C++}

\subsection{Inicjalizacje i sprawdzanie}

Minimalne przetwarzanie semantyczne odbywa się, gdy po raz pierwszy widzi definicję szablonu lub jego użycie. Pełne przetwarzanie semantyczne jest przesuwane na czas inicjalizacji (tuż przed czasem linkowania), na podstawie każdej instancji. Oznacza to, że założenia dotyczące argumentów szablonu nie są sprawdzane przed czasem inicjalizacji. Np.\newline

\noindent \verb#string x = "testowy tekst";# \newline
\verb#kwadrat(x);# \newline

Bezsensowne użycie zmiennej \verb#string# jako argumentu funkcji \verb#kwadrat# nie jest wyłapane w momencie użycia. Dopiero w czasie inicjalizacji kompilator odkryje, że nie ma odpowiedniej deklaracji dla operatora *. To ogromny praktyczny błąd, bo inicjalizacja może być przeprowadzona przez kod napisany przez użytkownika, który nie napisał definicji funkcji \verb#kwadrat# ani definicji \verb#string#. Programista, który nie znał definicji funkcji \verb#kwadrat# ani \verb#string# miałby ogromne trudności w zrozumieniu komunikatów błędów związanych z ich interakcją (np. ”illegal operand for *”).

Istnienie symbolu operatora * nie jest wystarczające by zapewnić pomyślną kompilację funkcji \verb#kwadrat#. Musi istnieć operator *, który przyjmuje argumenty odpowiednich typów i ten operator * musi być bezkonkurencyjnym dopasowaniem według zasad przeciążania C++. Dodatkowo funkcja \verb#kwadrat# przyjmuje argumenty przez wartość i zwraca swój wynik przez wartość. Z tego wynika, że musi być możliwe skopiowanie obiektów dedukowanego typu. Potrzebny jest rygorystyczny framework do opisywania wymagań definicji szablonów na ich argumentach.

Doświadczenie podpowiada nam, że pomyślna kompilacja i linkowanie może nie gwarantować końca problemów. Udana budowa pokazuje tylko, że inicjalizacje szablonów były poprawne pod względem typów, dostając argumenty które przekazaliśmy. Co z typami argumentów szablonu i wartościami, z którymi nie próbowaliśmy użyć naszych szablonów? Definicja szablonu może zawierać przypuszczenia na temat argumentów, które przekazaliśmy ale nie zadziała dla innych, prawdopodobnie rozsądnych argumentów. Uproszczona wersja klasycznego przykładu:

\begin{lstlisting}[frame=single]
template<typename FwdIter>
bool czyJestPalindromem(FwdIter first, FwdIter last){
   if(last <= first) return true;
   if(*first != *last) return false;
   return czyJestPalindromem(++first, --last);
}

\end{lstlisting}

Testujemy czy sekwencja wyznaczona przez parę iteratorów do jego pierwszego i ostatniego elementu, jest palindromem. Przyjmuje się, że te iteratory są z kategorii \emph{forward iterator}. To znaczy, że powinny wspierać co najmniej operacje takie jak: *, != i ++. Definicja funkcji \verb#czyJestPalindromem# bada czy elementy sekwencji zmierzają z początku i końca do środka. Możemy przetestować tę funkcję używając \verb#vector#, tablicę w stylu \verb#C# i \verb#string#. W każdym przypadku nasz szablon funkcji zainicjalizuje się i wykona się poprawnie. Niestety, umieszczenie tej funkcji w bibliotece byłoby dużym błędem. Nie wszystkie sekwencje wspierają \verb#--# i $\leq$. Np. listy pojedyncze nie wspierają. Eksperci używają wyszukanych, regularnych technik by uniknąć takich problemów.  Jednakże, fundamentalny problem jest taki, że definicja szablonu nie jest (według siebie) dobrą specyfikacją jego wymagań na jego parametry.

\end{document}

\documentclass[11pt, a4paper]{article}
\usepackage{polski}
\usepackage[utf8]{inputenc}
\usepackage{listings}

\begin{document}
\lstset{language=C++}

\subsection{Wydajność}

Szablony grają kluczową rolę w programowaniu w \emph{C++} dla wydajnych aplikacji. Ta wydajność ma trzy źródła:

\begin{itemize}

\item eliminacja wywołań funkcji na korzyść \emph{inliningu}
\item łączenie informacji z różnych kontekstów w celu lepszej optymalizacji
\item unikanie generowania kodu dla niewykorzystanych funkcji

\end{itemize}

Pierwszy punkt nie odnosi się tylko do szablonów ale ogólnie do cech funkcji \emph{inline} w \emph{C++}. Jakkolwiek, \emph{inlining} jest istotny dla drobno-granularnej parametryzacji, którą powszechnie stosuje się w bibliotece \emph{STL} i innych bibliotekach  bazujących na generycznych technikach programowania. Wydajność ta przekłada się zarówno na czas wykonania jak i pamięć. Szablony mogą równocześnie zmniejszyć obie wydajności. Zmniejszenie rozmiaru kodu jest szczególnie ważne, ponieważ w przypadku nowoczesnych procesorów zmniejszenie rozmiaru kodu pociąga za sobą zmniejszenie ruchu w pamięci i poprawienie wydajności pamięci podręcznej.

\begin{lstlisting}[frame=single]
template<typename FwdIter, typename T>
T suma(FwdIter first, FwdIter last, T init){
   for(FwdIter cur = first, cur != last, T init)
      init = init + *cur;
   return init;
}

\end{lstlisting}

Funkcja \verb#suma# zwraca sumę elementów jej sekwencji wejściowej używając trzeciego argumentu ("akumulatora") jako wartości początkowej\newline

\noindent \verb#vector<zespolona<double>> v;#  \newline
\verb#zespolona<double> z = 0;# \newline
\verb#z = suma(v.begin(), v.end(), z);# \newline

By wykonać swoją pracę, \verb#suma# użyje operatorów dodawania i przypisania na elementach typu \verb#zespolona<double># i dereferencji iteratorów \verb#vector<zespolona<double>>#.  Dodanie wartości typu \verb#zespolona<double># pociąga za sobą dodanie wartości typu \verb#double#. By zrobić to wydajnie wszystkie te operacje muszą być \emph{inline}.  Zarówno \verb#vector# jak i \verb#zespolona# są typami zdefiniowanymi przez użytkownika. Oznacza to, że typy te jak i ich operacje są zdefiniowane gdzie indziej w kodzie źródłowym \emph{C++}. Obecne kompilatory \emph{C++} radzą sobie z tym przykładem, dzięki czemu jedyne wygenerowane wywołanie to wywołanie funkcji \verb#suma#. Dostęp do pól zmiennej \verb#vector# staje się prostą operacją maszyny ładującej, dodawanie wartości typu \verb#zespolona# staje się dwiema instrukcjami maszyny dodającej dwa elementy zmiennoprzecinkowe. Aby to osiągnąć, kompilator potrzebuje dostępu do pełnej definicji \verb#vector# i \verb#zespolona#. Jednak wynik jest ogromną poprawą (prawdopodobnie optymalną) w stosunku do naiwnego podejścia generowania wywołania funkcji dla każdego użycia operacji na parametrze szablonu. Oczywiście instrukcja dodawania wykonuje się znacznie szybciej niż wywołanie funkcji zawierającej dodawanie. Poza tym, nie ma żadnego wstępu wywołania funkcji, przekazywanie argumentów itd., więc kod wynikowy jest również wiele mniejszy. Dalsze zmniejszanie rozmiaru generowanego kodu uzyskuje się nie wysyłając kodu niewykorzystywanych funkcji. Klasa szablonu \verb#vector# ma wiele funkcji, które nie są wykorzystywane w tym przykładzie. Podobnie szablon klasy \verb#zespolona# ma wiele funkcji i funkcji nieskładowych (nienależących do funkcji klasy). Standard \emph{C ++} gwarantuje, że nie jest emitowany żaden kod dla tych niewykorzystanych funkcji. 

Aby kontrastować, rozważ bardziej konwencjonalny przypadek, w którym argumenty są dostępne za pośrednictwem interfejsów zdefiniowanych jako wywołania funkcji pośrednich. Każda operacja staje się wtedy wywołaniem funkcji w pliku wykonywalnym generowanym dla kodu użytkownika, takiego jak suma. Co więcej, byłoby wyraźnie nietypowe unikać odkładania kodu nieużywanych (wirtualnych) funkcji składowych. Jest to poza zdolnością obecnych kompilatorów \emph{C++} i prawdopodobnie pozostanie takie dla głównych programów \emph{C++}, gdzie oddzielna kompilacja i łączenie dynamiczne jest normą. Ten problem nie jest wyjątkowy dla \emph{C++}. Opiera się on na podstawowej trudności w ocenieniu, która część kodu źródłowego jest używana, a która nie, gdy jakakolwiek forma wysyłki czasu wykonania ma miejsce. Szablony nie cierpią na ten problem bo ich specjalizacje są rozwiązywane w czasie kompilacji.

Przykład funkcji \verb#suma# nie jest idealny do zilustrowania subtelności generowania kodu obiektu z kodu źródłowego znalezionego w różnych częściach programu. Nie polega na niejawnych konwersjach lub nietypowych przeciążaniach. Jednak, rozważ wariant gdzie wartości int są sumowane w obiekcie \verb#zespolona<double>#: \newline

\noindent \verb#vector<int> v;#  \newline
\verb#zespolona<double> s = 0;#  \newline
\verb#s = suma(v.begin(), v.end(), s);#  \newline

Tu dodawanie jest wykonane przez konwertowanie wartości \verb#int# do wartości \verb#double# i potem dodawanie tego do akumulatora \verb#s#, używając operatora + typu \verb#zespolona<double># i \verb#double#. To podstawowe dodawanie zmiennoprzecinkowe. Kwestia jest taka, że operator + w funkcji \verb#suma# zależy od dwóch parametrów szablonu i leży to w kwestii kompilatora by wybrać bardziej odpowiedni operator + bazując na informacji o tych dwóch argumentach. Byłoby możliwe utrzymanie lepszego rozdzielenia między różnymi kontekstami przez zawsze przekształcanie typu elementu w typ akumulatora. W takim przypadku spowodowałoby to powstanie dodatkowego \verb#zespolona<double># dla każdego elementu i dodania dwóch wartości typu \verb#zespolona#. Rozmiar kodu i czas wykonywania byłyby większe niż dwukrotnie.

Nie spodziewalibyśmy się zobaczyć tego ostatniego przykładu bezpośrednio w kodzie źródłowym. Gdybyśmy go zobaczyli, uznaliśmy, że jest on źle napisany. Jednakże, równoważny kod jest powszechny w wyniku zagnieżdżonych abstrakcji. Jest to szczególnie ważne by generować dobry kod w takich przypadkach ponieważ nie robienie tego byłoby zniechęcające dla abstrakcji. 

Warto zauważyć, że te optymalizacje są wspólnym miejscem. Duże ilości prawdziwego oprogramowania zależą od nich. W konsekwencji udoskonalone sprawdzanie typu, co zostało obiecane przy użyciu konceptów, nie może kosztować tych optymalizacji.

\end{document}

\end{document}
	
	\newpage
	
	\documentclass[11pt, a4paper]{article}
\usepackage{polski}
\usepackage[utf8]{inputenc}
\usepackage{listings}
\usepackage{standalone}

\begin{document}
\lstset{language=C++}

\section{Koncepty}

\documentclass[11pt, a4paper]{article}
\usepackage{polski}
\usepackage[utf8]{inputenc}
\usepackage{listings}

\begin{document}
\lstset{language=C++}

\subsection*{Wprowadzenie}
W 1987 próbowano projektować szablony z odpowiednimi interfejsami. Chciano by szablony:

\begin{itemize}

\item były w pełni ogólne i wyraziste
\item by nie wykorzystywały większych zasobów w porównaniu do kodowania ręcznego
\item by miały dobrze określone interfejsy

\end{itemize}

\noindent Długo nie dało się osiągnąć tych trzech rzeczy, ale za to osiągnięto:

\begin{itemize}

\item \emph{kompletność Turinga}\footnote{(ang. Turing Completness) umiejętność do rozwiązania każdego zadania, czyli udzielenie odpowiedzi na każde zadanie. Program, który jest kompletny według Turinga może być wykorzystany do symulacji jakiejkolwiek 1-taśmowej maszyny Turinga}
\item lepszą wydajność (w porównaniu do kodu pisanego ręcznie)
\item kiepskie interfejsy (praktycznie \emph{typowanie kaczkowe czasu kompilacji})\footnote{(ang. duck typing) rozpoznanie typu obiektu, nie na podstawie deklaracji, ale przez badanie metod udostępnionych przez obiekt}

\end{itemize}

Brak dobrze określonych interfejsów prowadzi do spektakularnie złych wiadomości błędów. Dwie pozostałe właściwości uczyniły z szablonów sukces.

Rozwiązanie problemu specyfikacji interfejsu zostało, przez Alexa Stepanova nazwane \textbf{konceptami}. \textbf{Koncept} to zbiór wymagań argumentów szablonu. Można też go nazwać systemem typów dla szablonów, który obiecuje znacząco ulepszyć diagnostyki błędów i zwiększyć siłę ekspresji, taką jak przeciążanie oparte na konceptach oraz częściową specjalizację szablonu funkcji.

Koncepty (\emph{The Concepts TS}\footnote{(ang. The Concepts Technical Specification) Specyfikacja techniczna konceptów}) zostały opublikowane i zaimplementowane w wersji 6.1 kompilatora GCC w kwietniu 2016 roku. Fundamentalnie to predykaty czasu kompilacji typów i wartości. Mogą być łączone zwykłymi operatorami logicznymi (\verb#&&#, \verb#||#, \verb#!#)

\addcontentsline{toc}{subsection}{Wprowadzenie}

\end{document}

\documentclass[11pt, a4paper]{article}
\usepackage{polski}
\usepackage[utf8]{inputenc}
\usepackage{listings}

\begin{document}
\lstset{language=C++}

\subsection{Notacja skrótowa}
Gdy chcemy podkreślić, że argument szablonu ma być sekwencją, piszemy:

\begin{lstlisting}[frame=single]
template<typename Seq>
   requires Sequence<Seq>
void algo(Seq &s);

\end{lstlisting}

To oznacza, że potrzebujemy argumentu typu \verb#Seq#, który musi być typu \verb#Sequence#, lub innymi słowy: Szablon przyjmuje argument typu, który musi być typu \verb#Sequence#. Możemy to uprościć:

\begin{lstlisting}[frame=single]
template<Sequence Seq>
void algo(Seq &s);

\end{lstlisting}

To znaczy dokładnie to samo co dłuższa wersja, ale jest krótsza i lepiej wygląda. Używamy tej notacji dla konceptów z jednym argumentem. Np. moglibyśmy uprościć funkcję \verb#szukaj#: \newline

\begin{lstlisting}[frame=single]
template<Sequence S, typename T>
   requires Equality_comparable<Value_type<S>, T>
Iterator_of<S> szukaj(S &seq, const T &value);
\end{lstlisting}

Upraszcza to składnię języka. Sprawia, że nie jest zbyt zagmatwana.

\end{document}

\documentclass[11pt, a4paper]{article}
\usepackage{polski}
\usepackage[utf8]{inputenc}
\usepackage{listings}

\begin{document}
\lstset{language=C++}

\subsection{Notacja skrótowa}
Gdy chcemy podkreślić, że argument szablonu ma być sekwencją, piszemy:

\begin{lstlisting}[frame=single]
template<typename Seq>
   requires Sequence<Seq>
void algo(Seq &s);

\end{lstlisting}

To oznacza, że potrzebujemy argumentu typu \verb#Seq#, który musi być typu \verb#Sequence#, lub innymi słowy: Szablon przyjmuje argument typu, który musi być typu \verb#Sequence#. Możemy to uprościć:

\begin{lstlisting}[frame=single]
template<Sequence Seq>
void algo(Seq &s);

\end{lstlisting}

To znaczy dokładnie to samo co dłuższa wersja, ale jest krótsza i lepiej wygląda. Używamy tej notacji dla konceptów z jednym argumentem. Np. moglibyśmy uprościć funkcję \verb#szukaj#: \newline

\begin{lstlisting}[frame=single]
template<Sequence S, typename T>
   requires Equality_comparable<Value_type<S>, T>
Iterator_of<S> szukaj(S &seq, const T &value);
\end{lstlisting}

Upraszcza to składnię języka. Sprawia, że nie jest zbyt zagmatwana.

\end{document}

\documentclass[11pt, a4paper]{article}
\usepackage{polski}
\usepackage[utf8]{inputenc}
\usepackage{listings}

\begin{document}
\lstset{language=C++}

\subsection{Notacja skrótowa}
Gdy chcemy podkreślić, że argument szablonu ma być sekwencją, piszemy:

\begin{lstlisting}[frame=single]
template<typename Seq>
   requires Sequence<Seq>
void algo(Seq &s);

\end{lstlisting}

To oznacza, że potrzebujemy argumentu typu \verb#Seq#, który musi być typu \verb#Sequence#, lub innymi słowy: Szablon przyjmuje argument typu, który musi być typu \verb#Sequence#. Możemy to uprościć:

\begin{lstlisting}[frame=single]
template<Sequence Seq>
void algo(Seq &s);

\end{lstlisting}

To znaczy dokładnie to samo co dłuższa wersja, ale jest krótsza i lepiej wygląda. Używamy tej notacji dla konceptów z jednym argumentem. Np. moglibyśmy uprościć funkcję \verb#szukaj#: \newline

\begin{lstlisting}[frame=single]
template<Sequence S, typename T>
   requires Equality_comparable<Value_type<S>, T>
Iterator_of<S> szukaj(S &seq, const T &value);
\end{lstlisting}

Upraszcza to składnię języka. Sprawia, że nie jest zbyt zagmatwana.

\end{document}

\documentclass[11pt, a4paper]{article}
\usepackage{polski}
\usepackage[utf8]{inputenc}
\usepackage{listings}

\begin{document}
\lstset{language=C++}

\subsection{Notacja skrótowa}
Gdy chcemy podkreślić, że argument szablonu ma być sekwencją, piszemy:

\begin{lstlisting}[frame=single]
template<typename Seq>
   requires Sequence<Seq>
void algo(Seq &s);

\end{lstlisting}

To oznacza, że potrzebujemy argumentu typu \verb#Seq#, który musi być typu \verb#Sequence#, lub innymi słowy: Szablon przyjmuje argument typu, który musi być typu \verb#Sequence#. Możemy to uprościć:

\begin{lstlisting}[frame=single]
template<Sequence Seq>
void algo(Seq &s);

\end{lstlisting}

To znaczy dokładnie to samo co dłuższa wersja, ale jest krótsza i lepiej wygląda. Używamy tej notacji dla konceptów z jednym argumentem. Np. moglibyśmy uprościć funkcję \verb#szukaj#: \newline

\begin{lstlisting}[frame=single]
template<Sequence S, typename T>
   requires Equality_comparable<Value_type<S>, T>
Iterator_of<S> szukaj(S &seq, const T &value);
\end{lstlisting}

Upraszcza to składnię języka. Sprawia, że nie jest zbyt zagmatwana.

\end{document}

\documentclass[11pt, a4paper]{article}
\usepackage{polski}
\usepackage[utf8]{inputenc}
\usepackage{listings}

\begin{document}
\lstset{language=C++}

\subsection{Notacja skrótowa}
Gdy chcemy podkreślić, że argument szablonu ma być sekwencją, piszemy:

\begin{lstlisting}[frame=single]
template<typename Seq>
   requires Sequence<Seq>
void algo(Seq &s);

\end{lstlisting}

To oznacza, że potrzebujemy argumentu typu \verb#Seq#, który musi być typu \verb#Sequence#, lub innymi słowy: Szablon przyjmuje argument typu, który musi być typu \verb#Sequence#. Możemy to uprościć:

\begin{lstlisting}[frame=single]
template<Sequence Seq>
void algo(Seq &s);

\end{lstlisting}

To znaczy dokładnie to samo co dłuższa wersja, ale jest krótsza i lepiej wygląda. Używamy tej notacji dla konceptów z jednym argumentem. Np. moglibyśmy uprościć funkcję \verb#szukaj#: \newline

\begin{lstlisting}[frame=single]
template<Sequence S, typename T>
   requires Equality_comparable<Value_type<S>, T>
Iterator_of<S> szukaj(S &seq, const T &value);
\end{lstlisting}

Upraszcza to składnię języka. Sprawia, że nie jest zbyt zagmatwana.

\end{document}

\documentclass[11pt, a4paper]{article}
\usepackage{polski}
\usepackage[utf8]{inputenc}
\usepackage{listings}

\begin{document}
\lstset{language=C++}

\subsection{Notacja skrótowa}
Gdy chcemy podkreślić, że argument szablonu ma być sekwencją, piszemy:

\begin{lstlisting}[frame=single]
template<typename Seq>
   requires Sequence<Seq>
void algo(Seq &s);

\end{lstlisting}

To oznacza, że potrzebujemy argumentu typu \verb#Seq#, który musi być typu \verb#Sequence#, lub innymi słowy: Szablon przyjmuje argument typu, który musi być typu \verb#Sequence#. Możemy to uprościć:

\begin{lstlisting}[frame=single]
template<Sequence Seq>
void algo(Seq &s);

\end{lstlisting}

To znaczy dokładnie to samo co dłuższa wersja, ale jest krótsza i lepiej wygląda. Używamy tej notacji dla konceptów z jednym argumentem. Np. moglibyśmy uprościć funkcję \verb#szukaj#: \newline

\begin{lstlisting}[frame=single]
template<Sequence S, typename T>
   requires Equality_comparable<Value_type<S>, T>
Iterator_of<S> szukaj(S &seq, const T &value);
\end{lstlisting}

Upraszcza to składnię języka. Sprawia, że nie jest zbyt zagmatwana.

\end{document}

\documentclass[11pt, a4paper]{article}
\usepackage{polski}
\usepackage[utf8]{inputenc}
\usepackage{listings}

\begin{document}
\lstset{language=C++}

\subsection{Notacja skrótowa}
Gdy chcemy podkreślić, że argument szablonu ma być sekwencją, piszemy:

\begin{lstlisting}[frame=single]
template<typename Seq>
   requires Sequence<Seq>
void algo(Seq &s);

\end{lstlisting}

To oznacza, że potrzebujemy argumentu typu \verb#Seq#, który musi być typu \verb#Sequence#, lub innymi słowy: Szablon przyjmuje argument typu, który musi być typu \verb#Sequence#. Możemy to uprościć:

\begin{lstlisting}[frame=single]
template<Sequence Seq>
void algo(Seq &s);

\end{lstlisting}

To znaczy dokładnie to samo co dłuższa wersja, ale jest krótsza i lepiej wygląda. Używamy tej notacji dla konceptów z jednym argumentem. Np. moglibyśmy uprościć funkcję \verb#szukaj#: \newline

\begin{lstlisting}[frame=single]
template<Sequence S, typename T>
   requires Equality_comparable<Value_type<S>, T>
Iterator_of<S> szukaj(S &seq, const T &value);
\end{lstlisting}

Upraszcza to składnię języka. Sprawia, że nie jest zbyt zagmatwana.

\end{document}

\documentclass[11pt, a4paper]{article}
\usepackage{polski}
\usepackage[utf8]{inputenc}
\usepackage{listings}

\begin{document}
\lstset{language=C++}

\subsection{Notacja skrótowa}
Gdy chcemy podkreślić, że argument szablonu ma być sekwencją, piszemy:

\begin{lstlisting}[frame=single]
template<typename Seq>
   requires Sequence<Seq>
void algo(Seq &s);

\end{lstlisting}

To oznacza, że potrzebujemy argumentu typu \verb#Seq#, który musi być typu \verb#Sequence#, lub innymi słowy: Szablon przyjmuje argument typu, który musi być typu \verb#Sequence#. Możemy to uprościć:

\begin{lstlisting}[frame=single]
template<Sequence Seq>
void algo(Seq &s);

\end{lstlisting}

To znaczy dokładnie to samo co dłuższa wersja, ale jest krótsza i lepiej wygląda. Używamy tej notacji dla konceptów z jednym argumentem. Np. moglibyśmy uprościć funkcję \verb#szukaj#: \newline

\begin{lstlisting}[frame=single]
template<Sequence S, typename T>
   requires Equality_comparable<Value_type<S>, T>
Iterator_of<S> szukaj(S &seq, const T &value);
\end{lstlisting}

Upraszcza to składnię języka. Sprawia, że nie jest zbyt zagmatwana.

\end{document}

\end{document}
	
	\newpage
	
	%\documentclass[11pt, a4paper]{article}
\usepackage{polski}
\usepackage[utf8]{inputenc}
\usepackage{listings}
\usepackage{standalone}

\begin{document}
\lstset{language=C++}

\section{STL}

\documentclass[11pt, a4paper]{article}
\usepackage{polski}
\usepackage[utf8]{inputenc}
\usepackage{listings}

\begin{document}
\lstset{language=C++}

\subsection*{Wprowadzenie}
W 1987 próbowano projektować szablony z odpowiednimi interfejsami. Chciano by szablony:

\begin{itemize}

\item były w pełni ogólne i wyraziste
\item by nie wykorzystywały większych zasobów w porównaniu do kodowania ręcznego
\item by miały dobrze określone interfejsy

\end{itemize}

\noindent Długo nie dało się osiągnąć tych trzech rzeczy, ale za to osiągnięto:

\begin{itemize}

\item \emph{kompletność Turinga}\footnote{(ang. Turing Completness) umiejętność do rozwiązania każdego zadania, czyli udzielenie odpowiedzi na każde zadanie. Program, który jest kompletny według Turinga może być wykorzystany do symulacji jakiejkolwiek 1-taśmowej maszyny Turinga}
\item lepszą wydajność (w porównaniu do kodu pisanego ręcznie)
\item kiepskie interfejsy (praktycznie \emph{typowanie kaczkowe czasu kompilacji})\footnote{(ang. duck typing) rozpoznanie typu obiektu, nie na podstawie deklaracji, ale przez badanie metod udostępnionych przez obiekt}

\end{itemize}

Brak dobrze określonych interfejsów prowadzi do spektakularnie złych wiadomości błędów. Dwie pozostałe właściwości uczyniły z szablonów sukces.

Rozwiązanie problemu specyfikacji interfejsu zostało, przez Alexa Stepanova nazwane \textbf{konceptami}. \textbf{Koncept} to zbiór wymagań argumentów szablonu. Można też go nazwać systemem typów dla szablonów, który obiecuje znacząco ulepszyć diagnostyki błędów i zwiększyć siłę ekspresji, taką jak przeciążanie oparte na konceptach oraz częściową specjalizację szablonu funkcji.

Koncepty (\emph{The Concepts TS}\footnote{(ang. The Concepts Technical Specification) Specyfikacja techniczna konceptów}) zostały opublikowane i zaimplementowane w wersji 6.1 kompilatora GCC w kwietniu 2016 roku. Fundamentalnie to predykaty czasu kompilacji typów i wartości. Mogą być łączone zwykłymi operatorami logicznymi (\verb#&&#, \verb#||#, \verb#!#)

\addcontentsline{toc}{subsection}{Wprowadzenie}

\end{document}

\end{document}
	
	%\newpage
	
	%\section{Rozdział 4}
	%Zawartość rozdziału 4
	
	%\newpage

	%\section{Rozdział 5}
	%Zawartość rozdziału 5
	
	%\newpage
	
	\documentclass[11pt, a4paper]{article}
\usepackage{polski}
\usepackage[utf8]{inputenc}
\usepackage{listings}
\usepackage{standalone}

\begin{document}
\lstset{language=C++}

\section{Włączenie konceptów do standardu C++}

Koncepty nie zostały włączone do standardu C++17. Krótkie uzasadnienie jest takie, że komisja nie osiągnęła porozumienia, że koncepty (określone w specyfikacji technicznej) osiągnęły wystarczające doświadczenie w zakresie wdrożenia i użytkowania, aby być wystarczające do dopuszczenia w obecnym projekcie. Zasadniczo komisja nie powiedziała "nie", ale "jeszcze nie".

Największe zastrzeżenia nie wynikały z problemów technicznych. Powstały następujące obawy:

\begin{itemize}

\item specyfikacja konceptów istniała w opublikowanej formie przez mniej niż cztery miesiące
\item jedyna znana i dostępna publicznie implementacja konceptów znajduje się w nieopublikowanej wersji \emph/{kompilatora GCC}
\item implementacja kompilatora gcc została opracowana przez tę samą osobę, która napisała specyfikację. W związku z tym implementacja jest dostępna do testowania, ale nie podjęto żadnej próby wprowadzenia w życia specyfikacji. A zatem specyfikacja nie jest przetestowana. Kilku członków komisji wskazało, że posiadanie implementacji wyprodukowanej ze specyfikacji ma decydujące znaczenie dla określenia kwestii specyfikacji.
\item najbardziej znaczące i znane użycie konceptów jest dostępne w specyfikacji \emph{Ranges TS}\footnote{Ranges}. Jest kilka innych projektów eksperymentujących z konceptami, ale żaden z nich nie zbliża się do skali, której można oczekiwać gdy programiści zaczną korzystać z tej funkcjonalności. Wydajność i problemy związane z obsługą błędów przy użyciu bieżącej implementacji GCC dowodzą, że nie wykonano większej próby używania konceptów.
\item specyfikacja konceptów nie dostarcza żadnych definicji. Niektórzy członkowie komisji kwestionują użyteczność konceptów bez dostępności biblioteki definicji konceptów, takiej jak \emph{Ranges TS}. Przyjęcie specyfikacji konceptów do \emph{C++17} bez odpowiedniej biblioteki definicji niesie ryzyko zablokowania języka bez udowodnienia, że zawiera funkcje potrzebne do wdrożenia biblioteki, które mogłyby być zaprojektowane do konceptualizacji biblioteki standardowej.

\end{itemize}

Obawy techniczne:
\begin{itemize}

\item koncepty zawierają nową składnię do definiowania szablonów funkcji. Skrócona deklaracja szablonu funkcji wygląda podobnie to nieszablonowej deklaracji funkcji z wyjątkiem tego, że co najmniej jeden z jej  parametrów zostanie zadeklarowany ze specyfikatorem typu zastępczego \verb#auto# albo nazwą konceptu. Obawa wynika z tego, że taka deklaracja:\newline
\noindent \verb#void f(X x){}# \newline
definiuje nieszablonową funkcję jeśli \verb#X# jest typem, ale definiuje szablon funkcji jeśli \verb#X# jest konceptem. To ma subtelne konsekwencje dla tego czy funkcja może być zdefiniowana w pliku nagłówkowym, czy słowo kluczowe \verb#typename# jest potrzebne by odnieść się do składowych typów typu \verb#X#, czy istnieje dokładnie jedna zmienna lub brak lub kilka dla każdej deklarowanej zmiennej lokalnej, statycznej. itd.


\item specyfikacja konceptów zawiera również składnię szablonów wstępnych, która pozwala ominąć rozwlekłą składnię deklaracji szablonu, do której wszyscy są przyzwyczajeni jednocześnie określając ograniczenia typu. Następujący przykład deklaruje szablon funkcji \verb#f#, przyjmujący dwa parametry \verb#A# i \verb#B#, które spełniają wymagania konceptu \verb#C#: \newline
\verb#C{ A, B } void f(A a, B b);#\newline
Ta składnia nie jest lubiana. Wspomniano, że biblioteka \verb#Ranges TS# jej używała w pewnych miejscach a grupa pracująca nad ewolucją biblioteki zażądała żeby ją zmienić i już nigdy nie używać.
\item Są dwie formy definiowania konceptów: funkcja i zmienna. Forma funkcji istnieje po to by wspierać przeciążanie definicji konceptów oparte na parametrach szablonu. Forma zmiennej istnieje by wspierać nieco krótsze definicje:

\begin{lstlisting}[frame=single]
//forma funkcji
template<typename T>
concept bool C(){
   return ...;
}

//forma zmiennej
template<typename T>
concept bool C = ...;
\end{lstlisting}

Wszystkie koncepty, które można zdefiniować przy użyciu formy zmiennej można zdefiniować za pomocą formy funkcji. Stosowana forma wpływa na składnię wymaganą do oszacowania konceptu, a zatem użycie konceptu wymaga znajomości formy użytej do jego zdefiniowania. Wczesna wersja \emph{Ranges TS} używała zarówno formy zmiennej, jak i funkcji do definiowania konceptów i niespójność spowodowała wiele błędów w specyfikacji. Aktualna Ranges TS wykorzystuje tylko formę funkcji do zdefiniowania określonych konceptów. Niektórzy członkowie komitetu uważają, że jedna forma definicji uprości język i uniknie trudności w używaniu i nauczaniu. Zapewnienie odrębnej składni definicji konceptów, a nie określenie ich w kategoriach funkcji lub zmiennych uniknęłoby również dziwnej składni \verb#concept bool#.

\item została dodana możliwość używania \verb#auto# jako specyfikatora dla parametrów szablonu bez typu: \newline\newline
\verb#template<auto V>#\newline
\verb#constexpr auto v = V*2;#\newline

Z konceptami można by ograniczyć powyższy szablon tak, że typ \verb#V# spełniałby wymagania konceptu \verb#Integral#:\newline
\verb#template<Integral V>#\newline
\verb#constexpr auto v = V*2;#\newline

Jednak to jest ta sama składnia aktualnie używana przez \emph{Concepts TS}, do deklarowania parametrów typu szablonu ograniczonego. Jeśli \emph{Concepts TS} miały być wprowadzone, potrzebna by była inna składnia aby deklarować ograniczony parametr szablonu bez typu. Prawdopodobnie składnia stosowana przez Concepts TS bardziej nadaje się do deklarowania parametrów szablonów bez typu, jak pokazano powyżej, ponieważ pasuje do składni stosowanej dla innych deklaracji zmiennych. To oznacza, że nowa składnia do deklarowania ograniczonych parametrów typu byłaby pożądana ze względu na spójność języka.

\item Koncepty były powszechnie oczekiwane w celu uzyskania lepszych komunikatów o błędach niż obecnie są generowane, gdy pojawiają się niepowodzenie podczas tworzenia szablonów. Teoria idzie, ponieważ koncepty pozwalają odrzucić kod oparty na ograniczeniu w punkcie użycia szablonu, kompilator może po prostu zgłosić błąd ograniczenia, a nie błąd w niektórych wyrażeniach w potencjalnie głęboko zagnieżdżonym stosie instancji szablonu. Niestety okazuje się, że nie jest tak proste, a używanie konceptów skutkuje gorszymi komunikatami o błędach. Niepowodzenia ograniczeń często pojawiają się jako błędy w przeciążeniu, co powoduje potencjalnie długą listę kandydatów, z których każda ma własną listę przyczyn odrzucenia. Identyfikacja kandydata, który był przeznaczony do danego użycia, a następnie określenie, dlaczego wystąpiło niepowodzenia ograniczeń, może być gorszym doświadczeniem niż nawigowanie w stosie tworzenia instancji szablonów.

\item Wielu członków komisji wyraża zaniepokojenie faktem, czy obecny projekt konceptów wystarcza jako podstawa, na której można w przyszłości wdrożyć sprawdzenie pełnej definicji szablonu. Mimo zapewnień obrońców konceptów, że takie kontrole będą możliwe, wiele pytań pozostaje bez odpowiedzi, a członkowie komitetu pozostają bez przekonania. Wydaje się mało prawdopodobne, że obawy te zostaną rozwiązane w inny sposób niż poprzez wdrożenie sprawdzania definicji.

\end{itemize}

Wielu wierzy, że koncepty w jakiejś formie zostaną dodane do \emph{C++19/20}.

\end{document}
	
	\newpage
	
	\section{Bibliografia}
	\begin{thebibliography}{2}
	
	\bibitem{first} Gabriel Dos Reis, \emph{Generic Programming in C++: The Next Level.}, ACCU, 2002.
	\bibitem{second} Bjarne Stroustrup, \emph{The Design and Evolution of C++}, AddisonWesley, 1994
	\bibitem{third}  Bjarne Stroustrup, \emph{Expressing the standard library requirements asconcepts}	
	\bibitem{fourth} Gabriel Dos Reis, Bjarne Stroustrup, \href{http://www.stroustrup.com/popl06.pdf}{\emph{Specifying C++ Concepts}}
	\bibitem{fifth} J. C. Dehnert and A. Stepanov, \emph{Fundamentals of Generic Programming}, Dagstuhl Seminar on Generic Programming.1998. Springer LNCS.
	
	\end{thebibliography}

\end{document}