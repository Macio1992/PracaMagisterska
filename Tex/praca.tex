\documentclass[11pt, a4paper]{article}
\usepackage{polski}
\usepackage[utf8]{inputenc}

\author{Maciej Zbierowski}
\title{Koncepty jako sposób ograniczania argumentów szablonu}

\linespread{1.3}
\usepackage{hyperref}
\usepackage{listings}
\usepackage{standalone}

\hypersetup{
	colorlinks=true,
	linkcolor=black,
	filecolor=magenta,
	urlcolor=cyan,
	pdftitle=(Praca Magisterska)
	bookmarks=true,
	pdfpagemode=FullScreen
}

\begin{document}
	\pagenumbering{gobble}
	\maketitle
	\newpage
	\tableofcontents
	\pagenumbering{arabic}
	\newpage
	
	\section*{Wstęp}

	\documentclass[11pt, a4paper]{article}
\usepackage{polski}
\usepackage[utf8]{inputenc}
\usepackage{listings}
\usepackage{standalone}

\begin{document}
\lstset{language=C++}

Pomysł ograniczania argumentów szablonów jest tak stary jak stare są same szablony. Ale dopiero z początkiem dwudziestego pierwszego wieku zaczęły się poważne prace nad projektem języka C++, aby zapewnić te możliwości. Te prace ostatecznie dały rezultat w postaci \emph{konceptów C++0x}. Rozwój tych funkcjonalności i ich wdrożenie do biblioteki standardowej \emph{C++} były głównymi tematami \emph{Komisji Standardu C++}\footnote{(ang. C++ Standards Committee) znana również pod nazwą "ISO JTC1/SC22/WG21". Składa się z akredytowanych ekspertów z krajów członkowskich, którzy są zainteresowani pracą nad C++} dla C++11. Te cechy zostały ostatecznie usunięte z powodu istotnych nierozwiązanych kwestii i bardziej rygorystycznego terminu publikacji.

W 2010 roku wznowiono prace nad konceptami (bez udziału komisji). Andrew Sutton i Bjarne Stroustrup opublikowali dokument omawiający jak zminimalizować ilość konceptów potrzebnych do określania części biblioteki standardowej, a grupa z Uniwersytetu w Indianie zainicjowała prace nad nową implementacją. Potem wspólnie z Alexem Stepanovem (twórcą \emph{biblioteki STL}\footnote{(ang. Standard Template Library)}) stworzyli raport, w którymi przedstawili w pełni ograniczone algorytmy biblioteki STL i zasugerowali projekt języka, który mógłby wyrazić te ograniczenia. Próbowano zaprojektować minimalny zestaw funkcji językowych, które umożliwiłyby użytkownikom ograniczanie szablonów. Z tych prób narodziło się rozszerzenie języka, zwane \emph{Concepts Lite}.

Koncepty nie zostały włączone w \emph{C++17}. Niektórzy członkowie komisji uważali że nie minęło wystarczająco dużo czasu od publikacji specyfikacji technicznej, żeby sprawdzić czy projekt jest wystarczająco dobry, a wielu z nich było niezdecydowanych.

\end{document}

	\addcontentsline{toc}{section}{Wstęp}
	
	\documentclass[11pt, a4paper]{article}
\usepackage{polski}
\usepackage[utf8]{inputenc}
\usepackage{listings}

\begin{document}
\lstset{language=C++}

\section{Szablony - definicja, zastosowania}

Szablony są jedną z głównych cech C++. Dzięki nim możemy dostarczać generyczne typy i funkcje, 
bez kosztów czasu wykonania. Skupiają się na pisaniu kodu w sposób niezależny od konkretnego typu,
dzięki czemu wspierają programowanie generyczne. Grają kluczową rolę w projektowaniu obecnych, znanych
i popularnych bibliotek i systemów. Stanowią podstawę technik programowania w różnych dziedzinach, począwszy od konwencjonalnego programowania ogólnego przeznaczenia do oprogramowywania wbudowanych 
systemów bezpieczeństwa.

Szablon to coś w rodzaju przepisu, z którego translator C++ generuje deklaracje.

\begin{lstlisting}[frame=single]
template<typename T>
T kwadrat (T x) {
	return x * x;
}
\end{lstlisting}

Kod ten deklaruje rodzinę funkcji indeksowanych po parametrze typu. Można odnieść się do konkretnego członka tej rodziny przez zastosowanie konstrukcji \verb#kwadrat<int>#. Mówimy wtedy, że żądana jest specjalizacja szablonu dla funkcji \verb#kwadrat# z listą argumentów szablonu \verb#<int>#. Proces tworzenia specjalizacji nosi nazwę inicjalizacji szablonu, potocznie zwany inicjalizacją. Kompilator C++ stworzy odpowiedni odpowiednik definicji funkcji:

\begin{lstlisting}[frame=single]
int kwadrat(int x) {
	return x * x;
}
\end{lstlisting}

Argument typu int jest podstawiony za parametr typu T. Kod wynikowy jest sprawdzany pod względem typu, by zapewnić brak błędów wynikających z podmiany. Inicjalizacja szablonu jest wykonywana tylko raz dla danej specyfikacji nawet jeśli program zawiera jej wielokrotne żądania. 

W przeciwieństwie do języków takich jak Ada czy System F, lista argumentów szablonu może być pominięta z żądania inicjalizacji szablonu funkcji. Zazwyczaj, wartości parametrów szablonu są dedukowane.
\newline

\verb#double d = kwadrat(2.0);# \newline

Argument typu jest dedukowany na double. Warto zauważyć, że odmiennie niż w językach takich jak Haskell czy System F, parametry szablonu w C++ nie są ograniczone względem typów.

Szablonów używa się do zmniejszania kar abstrakcji i zjawiska \emph{code bloat} w systemach wbudowanych w stopniu, który jest niepraktyczny w standardowych systemach obiektowych. Robi się to z dwóch powodów:

\begin{itemize}

\item Po pierwsze, inicjalizacja szablonu łączy informacje zarówno z definicji, jak i z kontekstu użycia. To oznacza, że pełna informacja zarówno z definicji jak i z wywołanych kontekstów (włączając w to informacje o typach) jest udostępniana generatorowi kodu. Dzisiejsze generatory kodu dobrze sobie radzą z używaniem tych informacji w celu zminimalizowania czasu wykonania i przestrzeni kodu. Różni się to od zwykłego przypadku w języku obiektowym, gdzie wywołujący i wywoływany są kompletnie oddzieleni przez interfejs, który zakłada pośrednie wywołania funkcji.

\item Po drugie, szablon w C++ jest zazwyczaj domyślnie tworzony tylko jeśli jest używany w sposób niezbędny dla semantyki programu, automatycznie minimalizując miejsce w pamięci, które wykorzystuje aplikacja. W przeciwieństwie do języka Ada czy System F, gdzie programista musi wyraźnie zarządzać inicjalizacjami.

\end{itemize}

\subsection{Parametryzacja szablonów}

Parametry szablonu są specyfikowane na dwa sposoby:

\begin{enumerate}

\item \emph{parametry szablonu} – wyraźnie wspomniane jako parametry w deklaracji szablonu

\item \emph{nazwy zależne} - wywnioskowane z użycia parametrów w definicji szablonu

\end{enumerate}

W C++ nazwa nie może być użyta bez wcześniejszej deklaracji. To wymaga od użytkownika ostrożnego traktowania definicji szablonów. Np. w definicji funkcji \verb#kwadrat# nie ma widocznej deklaracji symbolu *. Jednak, podczas inicjalizacji szablonu \verb#kwadrat<int># kompilator może sprowadzić symbol * do (wbudowanego) operatora mnożenia dla wartości \verb#int#. Dla wywołania \verb#kwadrat(zespolona(2.0))#, operator * zostałby rozwiązany do (zdefiniowanego przez użytkownika) operatora mnożenia dla wartości \verb#zespolona#. Symbol * jest więc \emph{nazwą zależną} w definicji funkcji \verb#kwadrat#. Oznacza to, że jest to ukryty parametr definicji szablonu. Możemy uczynić z operacji mnożenia formalny parametr:

\begin{lstlisting}[frame=single]
template<typename Mul, typename T>
T square(T x) {
	return Mul() (x,x);
}
\end{lstlisting}

Pod-wyrażenie \verb#Mul()# tworzy obiekt funkcji, który wprowadza operacje mnożenia wartości typu \verb#T#. Pojęcie \emph{nazw zależnych} pomaga utrzymać liczbę jawnych argumentów.

\subsection{Inicjalizacje i sprawdzanie}

Minimalne przetwarzanie semantyczne odbywa się, gdy po raz pierwszy widzi definicję szablonu lub jego użycie. Pełne przetwarzanie semantyczne jest przesuwane na czas inicjalizacji (tuż przed czasem linkowania), na podstawie każdej instancji. Oznacza to, że założenia dotyczące argumentów szablonu nie są sprawdzane przed czasem inicjalizacji. Np.

\noindent \verb#string x = "testowy tekst";# \newline
\verb#kwadrat(x);#

Bezsensowne użycie string jako argumentu funkcji kwadrat nie jest wyłapane w momencie użycia. Dopiero w czasie inicjalizacji kompilator odkryje, że nie ma odpowiedniej deklaracji dla operatora *. To ogromny praktyczny błąd bo inicjalizacja może być przeprowadzona przez kod napisany przez użytkownika, który nie napisał definicji funkcji kwadrat ani definicji string. Programista, który nie znał definicji funkcji kwadrat ani string miałby ogromne trudności w zrozumieniu komunikatów błędów związanych z ich interakcją (np. ”illegal operand for *”).

Istnienie symbolu operatora * nie jest wystarczające by zapewnić pomyślną kompilację funkcji \verb#kwadrat#. Musi istnieć operator *, który przyjmuje argumenty odpowiednich typów i ten operator * musi być bezkonkurencyjnym dopasowaniem według zasad przeciążania C++. Dodatkowo funkcja \verb#kwadrat# przyjmuje argumenty przez wartość i zwraca swój wynik przez wartość. Z tego wynika, że musi być możliwe skopiowanie obiektów dedukowanego typu. Potrzebny jest rygorystyczny framework do opisywania wymagań definicji szablonów na ich argumentach.

Doświadczenie podpowiada nam, że pomyślna kompilacja i linkowanie może nie gwarantować końca problemów. Udana budowa pokazuje tylko, że inicjalizacje szablonów były poprawne pod względem typów, dostając argumenty które przekazaliśmy. Co z typami argumentów szablonu i wartościami, z którymi nie próbowaliśmy użyć naszych szablonów? Definicja szablonu może zawierać przypuszczenia na temat argumentów, które przekazaliśmy ale nie zadziała dla innych, prawdopodobnie rozsądnych argumentów. Uproszczona wersja klasycznego przykładu:

\begin{lstlisting}[frame=single]
template<typename FwdIter>
bool czyJestPalindromem(FwdIter first, FwdIter last){
	if(last <= first) return true;
	if(*first != *last) return false;
	return czyJestPalindromem(++first, --last);
}

\end{lstlisting}

Testujemy czy sekwencja wyznaczona przez parę iteratorów do jego pierwszego i ostatniego elementu, jest palindromem. Przyjmuje się, że te iteratory są z kategorii \emph{forward iterator}. To znaczy, że zakładają wspieranie co najmniej operacje: *, != i ++. Definicja funkcji \verb#czyJestPalindromem# bada czy elementy sekwencji zmierzają z początku i końca do środka. Możemy przetestować tę funkcję używając \verb#vector#, tablicą stylu \verb#C# i \verb#string#. W każdym przypadku nasz szablon funkcji zainicjalizuje się i wykona się poprawnie. Niestety, umieszczenie tej funkcji w bibliotece byłoby dużym błędem. Nie wszystkie sekwencje wspierają -- i <=. Np. listy pojedyncze nie wspierają. Eksperci używają wyszukanych, regularnych technik by uniknąć takich problemów.  Jednakże, fundamentalny problem jest taki że definicja szablonu nie jest (według siebie) dobrą specyfikacją jego wymagań na jego parametry.

\end{document}
	
	\newpage
	
	\documentclass[11pt, a4paper]{article}
\usepackage{polski}
\usepackage[utf8]{inputenc}
\usepackage{listings}

\begin{document}
\lstset{language=C++}

\section{Koncepty}

W 1987 próbowano projektować szablony z odpowiednimi interfejsami. Chciano by szablony:

\begin{itemize}

\item były w pełni ogólne i wyraziste
\item by nie wykorzystywały większych zasobów w porównaniu do kodowania ręcznego
\item by miały dobrze określone interfejsy

\end{itemize}

\noindent Długo nie dało się osiągnąć tych trzech rzeczy, ale za to osiągnięto:

\begin{itemize}

\item \emph{kompletność Turinga}\footnote{(ang. Turing Completness) umiejętność do rozwiązania każdej odpowiedzi. Program, który jest kompletny według Turinga może być wykorzystany do symulacji jakiejkolwiek 1-taśmowej maszyny Turinga}
\item lepszą wydajność (w porównaniu do kodu pisanego ręcznie)
\item kiepskie interfejsy (praktycznie \emph{typowanie kaczkowe czasu kompilacji})\footnote{(ang. duck typing) rozpoznanie typu obiektu, nie na podstawie deklaracji, ale przez badanie metod udostępnionych przez obiekt}

\end{itemize}

Brak dobrze określonych interfejsów prowadzi do spektakularnie złych wiadomości błędów. Dwie pozostałe właściwości uczyniły z szablonów sukces.

Rozwiązanie problemu specyfikacji interfejsu zostało, przez Alexa Stepanova nazwane \textbf{konceptami}. \textbf{Koncept} to zbiór wymagań argumentów szablonu. Można też go nazwać systemem typów dla szablonów, który obiecuje znacząco ulepszyć diagnostyki błędów i zwiększyć siłę ekspresji, taką jak przeciążanie oparte na konceptach oraz częściowa specjalizacja szablonu funkcji.

Koncepty (\emph{The Concepts TS}\footnote{(ang. The Concepts Technical Specification) Specyfikacja techniczna konceptów}) zostały opublikowane i zaimplementowane w wersji 6.1 kompilatora GCC w kwietniu 2016 roku. Fundamentalnie to predykaty czasu kompilacji typów i wartości. Mogą być łączone zwykłymi operatorami logicznymi (\verb#&&#, \verb#||#, \verb#!#)

\subsection{Podstawy konceptów}

Reprezentacja definicji szablonu w C++ to zazwyczaj drzewa wyprowadzania\footnote{(ang. Parse Trees)}. Używając identycznych technik kompilatora, możemy przekonwertować koncepty do drzew wyprowadzania. Posiadając to możemy zaimplementować sprawdzanie konceptów jako abstrakcyjne drzewo dopasowań\footnote{(ang. Abstract Tree Matching)}. Wygodnym sposobem implementowania takiego dopasowywania jest generowanie i porównywanie zestawów wymaganych funkcji i typów (zwane \emph{zestawami ograniczeń}) z definicji szablonów i konceptów.

Definicja konceptu to zestaw równań \emph{drzewa AST}\footnote{(ang. Abstract Syntax Tree)} z założeniami typu. Koncepty dają dwa zamysły:

\begin{enumerate}

\item w \emph{definicjach szablonu}, koncepty działają jak reguły osądzania typowania. Jeśli \emph{drzewo AST} zależy od parametrów szablonu i nie może być rozwiązane przez otaczające środowisko typowania, wtedy musi się pojawić w strzegących ciałach konceptów. Takie zależne \emph{drzewa AST} są domniemanymi parametrami konceptów i zostaną rozwiązane przez sprawdzanie konceptów w momentach użycia.

\item w \emph{użyciach szablonów}, koncepty działają jak zestawy predykatów, które argumenty szablonu muszą spełniać. Sprawdzanie konceptów rozwiązuje domniemane parametry w momentach inicjalizacji.

\end{enumerate}

\subsection{Ulepszenie programowania generycznego}
\verb#double pierwiastek(double d);# \newline
\verb#double d = 7;# \newline
\verb#double d2 = pierwiastek(d);# \newline
\verb#vector<string> v = {"jeden", "dwa"};# \newline
\verb#double d3 = pierwiastek(v);# \newline

Mamy funkcję \verb#pierwiastek#, która jako parametr przyjmuje zmienną typu \verb#double#. Jeśli dostarczymy taki typ, wszystko będzie w porządku, ale jeśli damy inny typ od razu otrzymamy pomocną wiadomość błędu.

\begin{lstlisting}[frame=single]
template<class T>
void sortuj(T &c){
   //kod sortowania
}
\end{lstlisting}

Kod funkcji \verb#sortuj# zależy od różnych właściwości typu \verb#T#, takiej jak posiadanie operatora [] \newline

\noindent \verb#vector<string> v = {"jeden", "dwa"};# \newline
\verb#sortuj(v);# \newline
\verb#//OK: zmienna v ma wszystkie syntaktyczne właściwości# \newline
\verb#wymagane przez funkcję sort# \newline\newline
\verb#double d = 7;# \newline
\verb#sortuj(d);# \newline
\verb#//Błąd: zmienna d nie ma operatora []#\newline

\noindent Mamy kilka problemów:\newline
\begin{itemize}

\item wiadomośc błędu jest niejednoznaczna i daleko jej do precyzyjnej i pomocnej, tak jak : "Błąd: zmienna d nie ma operatora []"

\item aby użyć funkcji \verb#sortuj#, musimy dostarczyć jej definicję, a nie tylko deklaracje. Jest to różnica w sposobie pisania zwykłego kodu i zmienia się model organizowania kodu

\item wymagania funkcji dotyczące typu argumentu są domniemane w ciałach ich funkcji

\item wiadomość błędu funkcji pojawi się tylko podczas inicjalizacji szablonu, a to może się zdarzyć bardzo długo po momencie wywołania

\item Notacja \verb#template<typename T># jest powtarzalna, bardzo nielubiana.

\end{itemize}

Używając konceptu, możemy dotrzeć do źródła problemu, poprzez poprawne określanie wymagań argumentów szablonu. Fragment kodu używającego konceptu \verb#Sortable#:\newline

\noindent \verb#void sortuj(Sortable &c);//(1)#\newline
\verb#vector<string> v = {"jeden", "dwa"};#\newline
\verb#sortuj(v);//(2)# \newline
\verb#double d = 7;# \newline
\verb#sortuj(d);//(3)# \newline

\noindent (1) - akceptuj jakąkolwiek zmienną c, która jest \verb#Sortable# \newline
(2) - OK: \verb#v# jest kontenerem typu \verb#Sortable# \newline
(3) - Błąd: d nie jest \verb#Sortable# (\verb#double# nie dostarcza operatora [], itd. \newline

Kod jest analogiczny do przykładu \verb#pierwiastek#. Jedyna różnica polega na tym, że:
\begin{itemize}

\item w przypadku typu \verb#double#, projektant języka wbudował go do kompilatora jak określony typ, gdzie jego znaczenie zostało określone w dokumentacji

\item zaś w przypadku \verb#Sortable#, użytkownik określił co on oznacza w kodzie. Typ jest \verb#Sortable# jeśli posiada właściwości \verb#begin()# i \verb#end()# dostarczające losowy dostęp do sekwencji zawierającej elementy, które mogą być porównywane używając operatora \verb#<#

\end{itemize}

Teraz otrzymujemy bardziej jasny komunikat błędu. Jest on generowany natychmiast w momencie gdzie kompilator widzi błędne wywołanie (\verb#sortuj(d);#)

Cele to zrobienie:
\begin{itemize}

\item kodu generycznego tak prostym jak nie-generyczny

\item bardziej zaawansowanego kodu generycznego tak łatwym do użycia i nie tak trudnym do
pisania

\end{itemize}

\subsection{Używanie konceptów}

Koncept to predykat czasu kompilacji (coś co zwraca wartość boolowską). Np. argument typu szablonu \verb#T# mógłby mieć wymagania żeby być:

\begin{itemize}

\item iteratorem \verb#Iterator<T>#

\item iteratorem losowego dostępu \verb#Random_access_iterator<T>#

\item liczbą: \verb#Number<T>#

\end{itemize}

Notacja \verb#C<T>#, gdzie \verb#C# to koncept a \verb#T# to typ, to wyrażenie znaczące "prawda jeśli \verb#T# spełnia wszystkie wymagania C, a nieprawda w przeciwnym wypadku."

Podobnie, możemy określić, że zestaw argumentów szablonu musi spełniać predykat, np. \verb#Mergeable<In1, In2, Out>#. Taki predykaty wielu typów są niezbędne do opisywania biblioteki STL i wielu innych. Są bardzo ekspresywne i łatwo kompilowalne (tańsze niż obejścia metaprogramowania szablonów). Można oczywiście definiować własne koncepty i można tworzyć biblioteki konceptów. Koncepty pozwalają na przeciążanie i eliminują potrzebę wielokrotnego doraźnego metaprogramowania i kodu \emph{scaffoldingu}\footnote{metaprogramistyczna metoda budowania aplikacji bazodanowych. To technika wspierana przez niektóre frameworki MVC, w których programista może napisać specyfikację opisującą sposób wykorzystania bazy danych aplikacji. Kompilator używa tej specyfikacji, aby wygenerować kod, który aplikacja może wykorzystać do odczytu, tworzenia, aktualizacji i usuwania wpisów bazy danych } z metaprogramowania, co znacznie upraszcza metaprogramowanie, a także programowanie generyczne.

\subsection{Określanie interfejsu szablonu}

\begin{lstlisting}[frame=single]
template<typename S, typename T>
   requires Sequence<S> && 
   Equality_comparable<Value_type<S>, T>
Iterator_of<S> szukaj(S &seq, const T &value);
\end{lstlisting}

Powyższy szablon przyjmuje dwa argumenty typu szablonu. Pierwszy argument typu musi być typu \verb#Sequence# i musimy być w stanie porównywać elementy sekwencji ze zmienną \verb#value# używając operatora \verb#==# (stąd \verb#Equality_comparable<Value_type<S>, T>#). Funkcja \verb#szukaj# przyjmuje sekwencję przez referencję i \verb#value# do znalezienia jako referencję \verb#const#. Zwraca iterator.

Sekwencja musi posiadać \verb#begin()# i \verb#end()#. Koncept \verb#Equality_comparable# jest zaproponowany jako koncept standardowej biblioteki. Wymaga by jego argument dostarczał operatory \verb#==# i \verb#!=#. Ten koncept przyjmuje dwa argumenty. Wiele konceptów przyjmuje więcej niż jeden argument. Koncepty mogą opisywać nie tylko typy, ale również związki między typami. \newline

Użycie funkcji \verb#szukaj#: \newline

\begin{lstlisting}[frame=single]
void test(vector<string> &v, list<double> &list){
   auto a0 = szukaj(v, "test");(1)
   auto p1 = szukaj(v, 0.7);(2)
   auto p2 = szukaj(list, 0.7);(3)
   auto p3 = szukaj(list, "test");(4)
   
   if(a0 != v.end()){
     //Znaleziono "test"
   }
}
\end{lstlisting}

1) OK
2) Błąd: nie można porównać string do double
3) OK
4) Błąd: nie można porównać double ze string

\subsection{Notacja skrótowa}
Gdy chcemy podkreślić, że argument szablonu ma być sekwencją, piszemy:

\begin{lstlisting}[frame=single]
template<typename Seq>
   requires Sequence<Seq>
void algo(Seq &s);

\end{lstlisting}

To oznacza, że potrzebujemy argumentu typu \verb#Seq#, który musi być typu \verb#Sequence#, lub innymi słowy: Szablon przyjmuje argument typu, który musi być typu \verb#Sequence#. Możemy to uprościć:

\begin{lstlisting}[frame=single]
template<Sequence Seq>
void algo(Seq &s);

\end{lstlisting}

To znaczy dokładnie to samo co dłuższa wersja, ale jest krótsza i lepiej wygląda. Używamy tej notacji dla konceptów z jednym argumentem. Np. moglibyśmy uprościć funkcję \verb#szukaj#: \newline

\begin{lstlisting}[frame=single]
template<Sequence S, typename T>
   requires Equality_comparable<Value_type<S>, T>
Iterator_of<S> szukaj(S &seq, const T &value);
\end{lstlisting}

Upraszcza to składnię języka. Sprawia, że nie jest zbyt zagmatwana.

\subsection{Definiowanie konceptów}

Koncepty, takie jak \verb#Equality_comparable# często można znaleźć w bibliotekach (np. w \verb#The Ranges TS#), ale koncepty można też definiować samodzielnie: \newline

\begin{lstlisting}[frame=single]
template<typename T>
concept bool Equality_comparable = requires (T a, T b){
   { a == b } -> bool; //(1)
   { a != b } -> bool; //(2)
};

\end{lstlisting}

Koncept ten został zdefiniowany jako szablonowa zmienna. Typ musi dostarczać operacje \verb#==# i \verb#!=#, z których każda musi zwracać wartość \verb#bool#, żeby być \verb#Equality_comparable#
. Wyrażenie \verb#requires# pozwala na bezpośrednie wyrażenie jak typ może być użyty:

\begin{itemize}

\item \verb#{ a == b }#, oznajmia, że dwie zmienne typu \verb#T# powinny być porównywalne używając operatora \verb#==#

\item \verb#{ a == b} -> bool# mówi że wynik takiego porównania musi być typu \verb#bool#

\end{itemize}

Wyrażenie \verb#requires# jest właściwie nigdy nie wykonywane. Zamiast tego kompilator patrzy na wymagania  i zwraca \verb#true# jeśli się skompilują a \verb#false# jeśli nie. To bardzo potężne ułatwienie. 

\begin{lstlisting}[frame=single]
template<typename T>
concept bool Sequence = requires(T t) {
   typename Value_type<T>;
   typename Iterator_of<T>;
   
   { begin(t) } -> Iterator_of<T>;
   { end(t) } -> Iterator_of<T>;
   
   requires Input_iterator<Iterator_of<T>>;
   requires Same_type<Value_type<T>,
   Value_type<Iterator_of<T>>>;
};

\end{lstlisting}

Żeby być typu \verb#Sequence#:

\begin{itemize}

\item typ \verb#T# musi mieć dwa powiązane typy: \verb#Value_type<T># i \verb#Iterator_of<T>#. Oba typy to zwykłe \emph{aliasy szablonu}\footnote{ALIAS SZABLONU}. Podanie tych typów w wyrażeniu \verb#requires# oznacza, że typ \verb#T# musi je posiadać żeby być \verb#Sequence#.

\item typ \verb#T# musi mieć operacje \verb#begin()# i \verb#end()#, które zwracają odpowiednie iteratory.

\item odpowiedni iterator oznacza to, że typ iteratora typu \verb#T# musi być typu \verb#Input_iterator# i typ wartości typu \verb#T# musi być taka sama jak jej wartość typu jej iteratora. \verb#Input_iterator# i \verb#Same_type# to koncepty z biblioteki.

\end{itemize}

Teraz w końcu możemy napisać koncept \verb#Sortable#. Żeby typ był \verb#Sortable#, powinien być sekwencją oferującą losowy dostęp i posiadać typ wartości, który wspiera porównania używające operatora \verb#<#:

\begin{lstlisting}[frame=single]
template<typename T>
concept bool Sortable = Sequence<T> &&
Random_access_iterator<Iterator_of<T>> &&
Less_than_comparable<Value_type<T>>;
\end{lstlisting}

\verb#Random_access_iterator# i \verb#Less_than_comparable# są zdefiniowane analogicznie do \verb#Equality_comparable#

Często, wymagane są relacje pomiędzy konceptami. Np. koncept \verb#Equality_comparable# jest zdefiniwoany by wymagał jeden typ. Można zdefiniować ten koncept by radził sobie z dwoma typami:

\begin{lstlisting}[frame=single]
template<typename T, typename U>
concept bool Equality_comparable = requires (T a, U b) {
   { a == b } -> bool;
   { a != b } -> bool;
   { b == a } -> bool;
   { b != a } -> bool;
};
\end{lstlisting}

To pozwala na porównywanie zmiennych typu \verb#int# z \verb#double# i \verb#string# z \verb#char*#, ale nie \verb#int# z \verb#string#.

\end{document}
	
	\newpage
	
	%\documentclass[11pt, a4paper]{article}
\usepackage{polski}
\usepackage[utf8]{inputenc}
\usepackage{listings}
\usepackage{standalone}

\begin{document}
\lstset{language=C++}

\section{STL}

\documentclass[11pt, a4paper]{article}
\usepackage{polski}
\usepackage[utf8]{inputenc}
\usepackage{listings}

\begin{document}

\subsection*{Wprowadzenie}

Każda szeroko używana biblioteka generyczna jest oparta na konceptach. Te koncepty można przedstawiać za pomocą specjalnie zaprojektowanych funkcji języka, tabel wymagań, komentarzy w kodzie, dokumentach projektu lub po prostu w głowach programistów. Lecz bez konceptów (formalnych czy nieformalnych), żaden kod generyczny by nie działał.

Podczas badań nad konceptami dochodzono do różnych wniosków. Zastanawiano się nad wpływem konceptów na niezmienniki (czy będą wzmocnione czy osłabną), opuszczaniem operatorów czy ograniczaniu przestrzeni definicji. Z tych rozważań wywnioskowano, że pisanie generycznej biblioteki wymaga wyczerpującego wyliczania przeładowanych operacji, wsparcia konceptów dla metaprogramowania szablonów i ogromnego bezpośredniego wsparcia języka dla konceptów.

Na szczęście, te podsumowanie mijają się z prawdą. Są po prostu złe. Przeważająca większość bibliotek generycznych, w tym oryginalna wersja biblioteki \emph{STL}\footnote{Standard Template Library}, została napisana bez wsparcia języka dla konceptów i używana bez mechanizmów wykonawczych. Podczas rozwoju bibliotek, ich autorzy nie rozważali setki różnych konceptów. Raczej, generyczne komponenty tych bibliotek zostały zapisane w małym zbiorze wyidealizowanych abstrakcji.

\addcontentsline{toc}{subsection}{Wprowadzenie}

\end{document}

\end{document}
	
	%\newpage
	
	%\section{Rozdział 4}
	%Zawartość rozdziału 4
	
	%\newpage

	%\section{Rozdział 5}
	%Zawartość rozdziału 5
	
	%\newpage
	
	\documentclass[11pt, a4paper]{article}
\usepackage{polski}
\usepackage[utf8]{inputenc}
\usepackage{listings}
\usepackage{standalone}

\begin{document}
\lstset{language=C++}

\section{Włączenie konceptów do standardu C++}

Koncepty nie zostały włączone do standardu C++17. Krótkie uzasadnienie jest takie, że komisja nie osiągnęła porozumienia, że koncepty (określone w specyfikacji technicznej) osiągnęły wystarczające doświadczenie w zakresie wdrożenia i użytkowania, aby być wystarczające do dopuszczenia w obecnym projekcie. Zasadniczo komisja nie powiedziała "nie", ale "jeszcze nie".

Największe zastrzeżenia nie wynikały z problemów technicznych. Powstały następujące obawy:

\begin{itemize}

\item specyfikacja konceptów istniała w opublikowanej formie przez mniej niż cztery miesiące
\item jedyna znana i dostępna publicznie implementacja konceptów znajduje się w nieopublikowanej wersji \emph/{kompilatora GCC}
\item implementacja kompilatora gcc została opracowana przez tę samą osobę, która napisała specyfikację. W związku z tym implementacja jest dostępna do testowania, ale nie podjęto żadnej próby wprowadzenia w życia specyfikacji. A zatem specyfikacja nie jest przetestowana. Kilku członków komisji wskazało, że posiadanie implementacji wyprodukowanej ze specyfikacji ma decydujące znaczenie dla określenia kwestii specyfikacji.
\item najbardziej znaczące i znane użycie konceptów jest dostępne w specyfikacji \emph{Ranges TS}\footnote{Ranges}. Jest kilka innych projektów eksperymentujących z konceptami, ale żaden z nich nie zbliża się do skali, której można oczekiwać gdy programiści zaczną korzystać z tej funkcjonalności. Wydajność i problemy związane z obsługą błędów przy użyciu bieżącej implementacji GCC dowodzą, że nie wykonano większej próby używania konceptów.
\item specyfikacja konceptów nie dostarcza żadnych definicji. Niektórzy członkowie komisji kwestionują użyteczność konceptów bez dostępności biblioteki definicji konceptów, takiej jak \emph{Ranges TS}. Przyjęcie specyfikacji konceptów do \emph{C++17} bez odpowiedniej biblioteki definicji niesie ryzyko zablokowania języka bez udowodnienia, że zawiera funkcje potrzebne do wdrożenia biblioteki, które mogłyby być zaprojektowane do konceptualizacji biblioteki standardowej.

\end{itemize}

Obawy techniczne:
\begin{itemize}

\item koncepty zawierają nową składnię do definiowania szablonów funkcji. Skrócona deklaracja szablonu funkcji wygląda podobnie to nieszablonowej deklaracji funkcji z wyjątkiem tego, że co najmniej jeden z jej  parametrów zostanie zadeklarowany ze specyfikatorem typu zastępczego \verb#auto# albo nazwą konceptu. Obawa wynika z tego, że taka deklaracja:\newline
\noindent \verb#void f(X x){}# \newline
definiuje nieszablonową funkcję jeśli \verb#X# jest typem, ale definiuje szablon funkcji jeśli \verb#X# jest konceptem. To ma subtelne konsekwencje dla tego czy funkcja może być zdefiniowana w pliku nagłówkowym, czy słowo kluczowe \verb#typename# jest potrzebne by odnieść się do składowych typów typu \verb#X#, czy istnieje dokładnie jedna zmienna lub brak lub kilka dla każdej deklarowanej zmiennej lokalnej, statycznej. itd.


\item specyfikacja konceptów zawiera również składnię szablonów wstępnych, która pozwala ominąć rozwlekłą składnię deklaracji szablonu, do której wszyscy są przyzwyczajeni jednocześnie określając ograniczenia typu. Następujący przykład deklaruje szablon funkcji \verb#f#, przyjmujący dwa parametry \verb#A# i \verb#B#, które spełniają wymagania konceptu \verb#C#: \newline
\verb#C{ A, B } void f(A a, B b);#\newline
Ta składnia nie jest lubiana. Wspomniano, że biblioteka \verb#Ranges TS# jej używała w pewnych miejscach a grupa pracująca nad ewolucją biblioteki zażądała żeby ją zmienić i już nigdy nie używać.
\item Są dwie formy definiowania konceptów: funkcja i zmienna. Forma funkcji istnieje po to by wspierać przeciążanie definicji konceptów oparte na parametrach szablonu. Forma zmiennej istnieje by wspierać nieco krótsze definicje:

\begin{lstlisting}[frame=single]
//forma funkcji
template<typename T>
concept bool C(){
   return ...;
}

//forma zmiennej
template<typename T>
concept bool C = ...;
\end{lstlisting}

Wszystkie koncepty, które można zdefiniować przy użyciu formy zmiennej można zdefiniować za pomocą formy funkcji. Stosowana forma wpływa na składnię wymaganą do oszacowania konceptu, a zatem użycie konceptu wymaga znajomości formy użytej do jego zdefiniowania. Wczesna wersja \emph{Ranges TS} używała zarówno formy zmiennej, jak i funkcji do definiowania konceptów i niespójność spowodowała wiele błędów w specyfikacji. Aktualna Ranges TS wykorzystuje tylko formę funkcji do zdefiniowania określonych konceptów. Niektórzy członkowie komitetu uważają, że jedna forma definicji uprości język i uniknie trudności w używaniu i nauczaniu. Zapewnienie odrębnej składni definicji konceptów, a nie określenie ich w kategoriach funkcji lub zmiennych uniknęłoby również dziwnej składni \verb#concept bool#.

\item została dodana możliwość używania \verb#auto# jako specyfikatora dla parametrów szablonu bez typu: \newline\newline
\verb#template<auto V>#\newline
\verb#constexpr auto v = V*2;#\newline

Z konceptami można by ograniczyć powyższy szablon tak, że typ \verb#V# spełniałby wymagania konceptu \verb#Integral#:\newline
\verb#template<Integral V>#\newline
\verb#constexpr auto v = V*2;#\newline

Jednak to jest ta sama składnia aktualnie używana przez \emph{Concepts TS}, do deklarowania parametrów typu szablonu ograniczonego. Jeśli \emph{Concepts TS} miały być wprowadzone, potrzebna by była inna składnia aby deklarować ograniczony parametr szablonu bez typu. Prawdopodobnie składnia stosowana przez Concepts TS bardziej nadaje się do deklarowania parametrów szablonów bez typu, jak pokazano powyżej, ponieważ pasuje do składni stosowanej dla innych deklaracji zmiennych. To oznacza, że nowa składnia do deklarowania ograniczonych parametrów typu byłaby pożądana ze względu na spójność języka.

\item Koncepty były powszechnie oczekiwane w celu uzyskania lepszych komunikatów o błędach niż obecnie są generowane, gdy pojawiają się niepowodzenie podczas tworzenia szablonów. Teoria idzie, ponieważ koncepty pozwalają odrzucić kod oparty na ograniczeniu w punkcie użycia szablonu, kompilator może po prostu zgłosić błąd ograniczenia, a nie błąd w niektórych wyrażeniach w potencjalnie głęboko zagnieżdżonym stosie instancji szablonu. Niestety okazuje się, że nie jest tak proste, a używanie konceptów skutkuje gorszymi komunikatami o błędach. Niepowodzenia ograniczeń często pojawiają się jako błędy w przeciążeniu, co powoduje potencjalnie długą listę kandydatów, z których każda ma własną listę przyczyn odrzucenia. Identyfikacja kandydata, który był przeznaczony do danego użycia, a następnie określenie, dlaczego wystąpiło niepowodzenia ograniczeń, może być gorszym doświadczeniem niż nawigowanie w stosie tworzenia instancji szablonów.

\item Wielu członków komisji wyraża zaniepokojenie faktem, czy obecny projekt konceptów wystarcza jako podstawa, na której można w przyszłości wdrożyć sprawdzenie pełnej definicji szablonu. Mimo zapewnień obrońców konceptów, że takie kontrole będą możliwe, wiele pytań pozostaje bez odpowiedzi, a członkowie komitetu pozostają bez przekonania. Wydaje się mało prawdopodobne, że obawy te zostaną rozwiązane w inny sposób niż poprzez wdrożenie sprawdzania definicji.

\end{itemize}

Wielu wierzy, że koncepty w jakiejś formie zostaną dodane do \emph{C++19/20}.

\end{document}
	
	\newpage
	
	\section{Bibliografia}
	\begin{thebibliography}{2}
	
	\bibitem{first} Gabriel Dos Reis, \emph{Generic Programming in C++: The Next Level.}, ACCU, 2002.
	\bibitem{second} Bjarne Stroustrup, \emph{The Design and Evolution of C++}, AddisonWesley, 1994
	\bibitem{third}  Bjarne Stroustrup, \emph{Expressing the standard library requirements asconcepts}	
	\bibitem{fourth} Gabriel Dos Reis, Bjarne Stroustrup, \href{http://www.stroustrup.com/popl06.pdf}{\emph{Specifying C++ Concepts}}
	\bibitem{fifth} J. C. Dehnert and A. Stepanov, \emph{Fundamentals of Generic Programming}, Dagstuhl Seminar on Generic Programming.1998. Springer LNCS.
	
	\end{thebibliography}

\end{document}