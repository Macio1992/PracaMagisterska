\documentclass[11pt, a4paper]{article}
\usepackage{polski}
\usepackage[utf8]{inputenc}
\usepackage{listings}

\begin{document}
\lstset{language=C++}

\subsection{Ulepszenie programowania generycznego}

Specyfikacja konceptów zawiera wiele ulepszeń, by lepiej wspierać programowanie generyczne przez:
\begin{itemize}

\item umożliwienie wyraźnego określenia ograniczeń argumentów szablonu jako części deklaracji szablonów
\item wsparcie możliwości przeciążania szablonów funkcji i częściowego określania szablonów klas i zmiennych opartych na tych ograniczeniach
\item dostarczenie składni do definiowania konceptów i wymagań narzuconych na argumenty szablonu
\item ujednolicenie \verb#auto# i konceptów w celu zapewnienia jednolitej i dostępnej notacji dla programowania ogólnego
\item radykalnej poprawy jakości wiadomości błędów wynikających z niewłaściwego wykorzystania szablonów
\item osiągnięcie powyższych bez żadnego narzucania jakichkolwiek zasobów ani znacznego wzrostu czasu kompilacji
\item bez ograniczania tego, co można wyrazić przy użyciu szablonów

\end{itemize}

\noindent\verb#double pierwiastek(double d);# \newline
\verb#double d = 7;# \newline
\verb#double d2 = pierwiastek(d);# \newline
\verb#vector<string> v = {"jeden", "dwa"};# \newline
\verb#double d3 = pierwiastek(v);# \newline

Mamy funkcję \verb#pierwiastek#, która jako parametr przyjmuje zmienną typu \verb#double#. Jeśli dostarczymy taki typ, wszystko będzie w porządku, ale jeśli damy inny typ od razu otrzymamy pomocną wiadomość błędu.

\begin{lstlisting}[frame=single]
template<class T>
void sortuj(T &c){
   //kod sortowania
}
\end{lstlisting}

Kod funkcji \verb#sortuj# zależy od różnych właściwości typu \verb#T#, takiej jak posiadanie operatora [] \newline

\noindent \verb#vector<string> v = {"jeden", "dwa"};# \newline
\verb#sortuj(v);# \newline
\verb#//OK: zmienna v ma wszystkie syntaktyczne właściwości# \newline
\verb#wymagane przez funkcję sort# \newline\newline
\verb#double d = 7;# \newline
\verb#sortuj(d);# \newline
\verb#//Błąd: zmienna d nie ma operatora []#\newline

\noindent Mamy kilka problemów:\newline
\begin{itemize}

\item wiadomość błędu jest niejednoznaczna i daleko jej do precyzyjnej i pomocnej, tak jak : "Błąd: zmienna d nie ma operatora []"

\item aby użyć funkcji \verb#sortuj#, musimy dostarczyć jej definicję, a nie tylko deklaracje. Jest to różnica w sposobie pisania zwykłego kodu i zmienia się model organizowania kodu

\item wymagania funkcji dotyczące typu argumentu są domniemane w ciałach ich funkcji

\item wiadomość błędu funkcji pojawi się tylko podczas inicjalizacji szablonu, a to może się zdarzyć bardzo długo po momencie wywołania

\item Notacja \verb#template<typename T># jest powtarzalna, bardzo nielubiana.

\end{itemize}

Używając konceptu, możemy dotrzeć do źródła problemu, poprzez poprawne określanie wymagań argumentów szablonu. Fragment kodu używającego konceptu \verb#Sortable#:\newline

\noindent \verb#void sortuj(Sortable &c);//(1)#\newline
\verb#vector<string> v = {"jeden", "dwa"};#\newline
\verb#sortuj(v);//(2)# \newline
\verb#double d = 7;# \newline
\verb#sortuj(d);//(3)# \newline

\noindent (1) - akceptuj jakąkolwiek zmienną c, która jest \verb#Sortable# \newline
(2) - OK: \verb#v# jest kontenerem typu \verb#Sortable# \newline
(3) - Błąd: d nie jest \verb#Sortable# (\verb#double# nie dostarcza operatora [], itd. \newline

Kod jest analogiczny do przykładu \verb#pierwiastek#. Jedyna różnica polega na tym, że:
\begin{itemize}

\item w przypadku typu \verb#double#, projektant języka wbudował go do kompilatora jak określony typ, gdzie jego znaczenie zostało określone w dokumentacji

\item zaś w przypadku \verb#Sortable#, użytkownik określił co on oznacza w kodzie. Typ jest \verb#Sortable# jeśli posiada właściwości \verb#begin()# i \verb#end()# dostarczające losowy dostęp do sekwencji zawierającej elementy, które mogą być porównywane używając operatora \verb#<#

\end{itemize}

Teraz otrzymujemy bardziej jasny komunikat błędu. Jest on generowany natychmiast w momencie gdzie kompilator widzi błędne wywołanie (\verb#sortuj(d);#)

Cele konceptów to, zrobienie:
\begin{itemize}

\item kodu generycznego tak prostym jak nie-generyczny

\item bardziej zaawansowanego kodu generycznego tak łatwym do użycia i nie tak trudnym do
pisania

\end{itemize}

\end{document}