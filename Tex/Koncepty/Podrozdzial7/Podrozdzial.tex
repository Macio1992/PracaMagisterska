\documentclass[11pt, a4paper]{article}
\usepackage{polski}
\usepackage[utf8]{inputenc}
\usepackage{listings}

\begin{document}
\lstset{language=C++}

\subsection{Definiowanie konceptów}

Koncepty, takie jak \verb#Equality_comparable# często można znaleźć w bibliotekach (np. w \verb#The Ranges TS#), ale koncepty można też definiować samodzielnie: \newline

\begin{lstlisting}[frame=single]
template<typename T>
concept bool Equality_comparable = requires (T a, T b){
   { a == b } -> bool; //(1)
   { a != b } -> bool; //(2)
};

\end{lstlisting}

Koncept ten został zdefiniowany jako szablonowa zmienna. Typ musi dostarczać operacje \verb#==# i \verb#!=#, z których każda musi zwracać wartość \verb#bool#, żeby być \verb#Equality_comparable#
. Wyrażenie \verb#requires# pozwala na bezpośrednie wyrażenie jak typ może być użyty:

\begin{itemize}

\item \verb#{ a == b }#, oznajmia, że dwie zmienne typu \verb#T# powinny być porównywalne używając operatora \verb#==#

\item \verb#{ a == b} -> bool# mówi że wynik takiego porównania musi być typu \verb#bool#

\end{itemize}

Wyrażenie \verb#requires# jest właściwie nigdy nie wykonywane. Zamiast tego kompilator patrzy na wymagania  i zwraca \verb#true# jeśli się skompilują a \verb#false# jeśli nie. To bardzo potężne ułatwienie. 

\begin{lstlisting}[frame=single]
template<typename T>
concept bool Sequence = requires(T t) {
   typename Value_type<T>;
   typename Iterator_of<T>;
   
   { begin(t) } -> Iterator_of<T>;
   { end(t) } -> Iterator_of<T>;
   
   requires Input_iterator<Iterator_of<T>>;
   requires Same_type<Value_type<T>,
   Value_type<Iterator_of<T>>>;
};

\end{lstlisting}

Żeby być typu \verb#Sequence#:

\begin{itemize}

\item typ \verb#T# musi mieć dwa powiązane typy: \verb#Value_type<T># i \verb#Iterator_of<T>#. Oba typy to zwykłe \emph{aliasy szablonu}\footnote{ALIAS SZABLONU}. Podanie tych typów w wyrażeniu \verb#requires# oznacza, że typ \verb#T# musi je posiadać żeby być \verb#Sequence#.

\item typ \verb#T# musi mieć operacje \verb#begin()# i \verb#end()#, które zwracają odpowiednie iteratory.

\item odpowiedni iterator oznacza to, że typ iteratora typu \verb#T# musi być typu \verb#Input_iterator# i typ wartości typu \verb#T# musi być taka sama jak jej wartość typu jej iteratora. \verb#Input_iterator# i \verb#Same_type# to koncepty z biblioteki.

\end{itemize}

Teraz w końcu możemy napisać koncept \verb#Sortable#. Żeby typ był \verb#Sortable#, powinien być sekwencją oferującą losowy dostęp i posiadać typ wartości, który wspiera porównania używające operatora \verb#<#:

\begin{lstlisting}[frame=single]
template<typename T>
concept bool Sortable = Sequence<T> &&
Random_access_iterator<Iterator_of<T>> &&
Less_than_comparable<Value_type<T>>;
\end{lstlisting}

\verb#Random_access_iterator# i \verb#Less_than_comparable# są zdefiniowane analogicznie do \verb#Equality_comparable#

Często, wymagane są relacje pomiędzy konceptami. Np. koncept \verb#Equality_comparable# jest zdefiniwoany by wymagał jeden typ. Można zdefiniować ten koncept by radził sobie z dwoma typami:

\begin{lstlisting}[frame=single]
template<typename T, typename U>
concept bool Equality_comparable = requires (T a, U b) {
   { a == b } -> bool;
   { a != b } -> bool;
   { b == a } -> bool;
   { b != a } -> bool;
};
\end{lstlisting}

To pozwala na porównywanie zmiennych typu \verb#int# z \verb#double# i \verb#string# z \verb#char*#, ale nie \verb#int# z \verb#string#.

\end{document}