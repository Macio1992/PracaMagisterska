\documentclass[11pt, a4paper]{article}
\usepackage{polski}
\usepackage[utf8]{inputenc}
\usepackage{listings}

\begin{document}
\lstset{language=C++}

\subsection{Określanie interfejsu szablonu}

\begin{lstlisting}[frame=single]
template<typename S, typename T>
   requires Sequence<S> && 
   Equality_comparable<Value_type<S>, T>
Iterator_of<S> szukaj(S &seq, const T &value);
\end{lstlisting}

Powyższy szablon przyjmuje dwa argumenty typu szablonu. Pierwszy argument typu musi być typu \verb#Sequence# i musimy być w stanie porównywać elementy sekwencji ze zmienną \verb#value# używając operatora \verb#==# (stąd \verb#Equality_comparable<Value_type<S>, T>#). Funkcja \verb#szukaj# przyjmuje sekwencję przez referencję i \verb#value# do znalezienia jako referencję \verb#const#. Zwraca iterator.

Sekwencja musi posiadać \verb#begin()# i \verb#end()#. Koncept \verb#Equality_comparable# jest zaproponowany jako koncept standardowej biblioteki. Wymaga by jego argument dostarczał operatory \verb#==# i \verb#!=#. Ten koncept przyjmuje dwa argumenty. Wiele konceptów przyjmuje więcej niż jeden argument. Koncepty mogą opisywać nie tylko typy, ale również związki między typami. \newline

Użycie funkcji \verb#szukaj#: \newline

\begin{lstlisting}[frame=single]
void test(vector<string> &v, list<double> &list){
   auto a0 = szukaj(v, "test");(1)
   auto p1 = szukaj(v, 0.7);(2)
   auto p2 = szukaj(list, 0.7);(3)
   auto p3 = szukaj(list, "test");(4)
   
   if(a0 != v.end()){
     //Znaleziono "test"
   }
}
\end{lstlisting}

1) OK
2) Błąd: nie można porównać string do double
3) OK
4) Błąd: nie można porównać double ze string

\end{document}