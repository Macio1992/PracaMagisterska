\documentclass[11pt, a4paper]{article}
\usepackage{polski}
\usepackage[utf8]{inputenc}
\usepackage{listings}

\begin{document}
\lstset{language=C++}

\subsection{Notacja skrótowa}
Gdy chcemy podkreślić, że argument szablonu ma być sekwencją, piszemy:

\begin{lstlisting}[frame=single]
template<typename Seq>
   requires Sequence<Seq>
void algo(Seq &s);

\end{lstlisting}

To oznacza, że potrzebujemy argumentu typu \verb#Seq#, który musi być typu \verb#Sequence#, lub innymi słowy: Szablon przyjmuje argument typu, który musi być typu \verb#Sequence#. Możemy to uprościć:

\begin{lstlisting}[frame=single]
template<Sequence Seq>
void algo(Seq &s);

\end{lstlisting}

To znaczy dokładnie to samo co dłuższa wersja, ale jest krótsza i lepiej wygląda. Używamy tej notacji dla konceptów z jednym argumentem. Np. moglibyśmy uprościć funkcję \verb#szukaj#: \newline

\begin{lstlisting}[frame=single]
template<Sequence S, typename T>
   requires Equality_comparable<Value_type<S>, T>
Iterator_of<S> szukaj(S &seq, const T &value);
\end{lstlisting}

Upraszcza to składnię języka. Sprawia, że nie jest zbyt zagmatwana.

\end{document}